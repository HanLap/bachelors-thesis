\documentclass[
  aspectratio=169
  ]{beamer}
\mode<presentation>
{
  \usetheme{myulm}
  \setbeamercovered{transparent}
  \setbeamertemplate{navigation symbols}{} % no navigation bar
  \setbeamersize{sidebar width left=1.17cm}
}

\usepackage[english]{babel}
\usepackage[utf8]{inputenc}
\usepackage{amsmath,amssymb,amsfonts}
\usepackage{times}
\usepackage{graphicx}
\usepackage{fancyvrb}
\usepackage{array}
\usepackage{colortbl} % ING INF
\usepackage{arydshln}
\usepackage{listings}
\usepackage{url}
\usepackage{hyperref}
\usepackage[inkscape=newer, inkscapelatex=false, inkscapearea=page]{svg} 

% Literaturverzeichnis mit BibLaTeX // use Biber as Backend; dashed = false to repeat author names
\usepackage[babel]{csquotes}
\usepackage[backend=biber,style=ieee,dashed=false,hyperref,natbib]{biblatex}
\addbibresource{BA.bib}


\usepackage{pgfplots}

\pgfkeys{/pgf/number format/.cd,1000 sep={}}
% !ignore-section
\pgfplotsset{
	compat=newest,
	legend style={at={(0.5,-0.2)},anchor=north}, % legend to the bottom
	legend cell align={left}, % left align legend text
	ymajorgrids=true, 
	grid style=dashed,
	scaled ticks=false, % don't use 10^x notation
	tick label style={/pgf/number format/fixed},
	try min ticks=8,
	tick pos=left % no ticks at the right / top borders
}
% !/ignore-section


% Anfang der Titelfolie
% Anpassung von: Titel, Untertitel, Autor, Datum und Institut

\title{BA Abschlussvortrag}
\subtitle{Development and Evaluation of a Metamodel \\ to Define Modeling Syntaxes for CouchEdit}
\author{Hannah Lappe}
\newcommand{\presdatum}{Oktober 17, 2020} % alternativ zu \today: Eingabe eines festen Datums
\institute
{\\Institut für Softwaretechnik und Programmiersprachen}
%Ende der Titelfolie


% Anfang der Kopfzeile der Folien
% Anpassung von: Zwischentitel, Leitthema oder Name
% Das Datum wird oben geändert: unter \presdatum{}!

\newcommand{\zwischentitel}{\insertsection}
\newcommand{\leitthema}{CouchEdit Metamodel}
% Ende der Kopfzeile

\newcommand{\red}[1]{\textcolor{farbwert-inwiin}{#1}}

% Anfang der Folien
\begin{document}

\hspace*{-1.49cm}
\frame[plain]{\titlepage}

% Das Inhaltsverzeichnis
% \hspace*{-0.7cm}
% \section*{Contents} % diese Section erscheint nicht im Inhaltsverzeichnis
% \begin{frame}
%   \frametitle{Table of Contents}
%   \tableofcontents
% \end{frame}

\section{What is CouchEdit?}

\begin{frame}
  \frametitle{What is CouchEdit?}
  \hspace{-1cm}
  \includegraphics[width=10cm]{images/"editor"}
\end{frame}

% \begin{frame}
%   \frametitle{Tight Coupling}
%   \hspace*{-10pt}\makebox[\linewidth][c]{
%     \begin{tabular}{c|c}
%       \includegraphics[width=5cm,trim={1cm 1cm 0 0},clip]{images/"presentation - concrete-instance"}
%        &
%       \includegraphics[width=5cm,trim={0 1cm 1cm 0},clip]{images/"presentation - abstract-instance"}
%       \\
%       Concrete Syntax Instance
%        &
%       Abstract Syntax Instance
%     \end{tabular}
%   }
% \end{frame}

% \begin{frame}
%   \frametitle{Tight Coupling}
%   \hspace{-1.5cm}
%   \includegraphics[width=11cm]{images/"presentation - thight-couple"}
% \end{frame}

% \begin{frame}
%   \centering
%   \hspace{-1cm}
%   \huge
%   \red{What is CouchEdit?}
% \end{frame}

% \begin{frame}
%   \frametitle{Tight Coupling}
%   \hspace{3cm}

%   \hspace*{-10pt}\makebox[\linewidth][c]{
%     \begin{tabular}{c|c}
%       \includegraphics[width=5cm,trim={1cm 1cm 0 1cm},clip]{images/"presentation - concrete-metamodel"}
%        &
%       \includegraphics[width=5cm,trim={0 1cm 1cm 1cm},clip]{images/"presentation - abstract-metamodel"}
%       \\
%       Render Metamodel
%        &
%       Abstract Syntax Metamodel
%       \\
%       \hdashline
%       \includegraphics[width=5cm,trim={1cm 1cm 0 0},clip]{images/"presentation - concrete-instance"}
%        &
%       \includegraphics[width=5cm,trim={0 1cm 1cm 0},clip]{images/"presentation - abstract-instance"}
%       \\
%       Concrete Syntax Instance
%        &
%       Abstract Syntax Instance
%     \end{tabular}
%   }
% \end{frame}

\begin{frame}
  \frametitle{}
  \hspace{-1.5cm}
  \includegraphics[width=11cm]{images/"presentation - transfere-metamodel"}
\end{frame}

% \begin{frame}
%   \frametitle{connection}
%   \centering
%   \hspace{-1cm}
%   \includegraphics[width=10cm]{images/"presentation - translation"}
% \end{frame}

\begin{frame}
  \frametitle{Hypergraph}
  \hspace{-1cm}
  \includegraphics[width=10cm]{images/"presentation - hypergraph"}
\end{frame}

\begin{frame}
  \frametitle{Processors}
  \hspace{-1cm}
  \includegraphics[width=10cm]{images/"component - bus"}
\end{frame}
\section{Problem Statement}
\begin{frame}
  \frametitle{}
  \centering
  \hspace{-1cm}
  \huge
  \red{Problem Statement}
\end{frame}


\begin{frame}
  \frametitle{Adding Language Support}
  current:
  \begin{itemize}
    \item high LoC count
    \item framework knowledge required
  \end{itemize}
\end{frame}

\begin{frame}
  \frametitle{Purpose of this Study}
  \vspace{-1cm}
  \onslide<1->{Development of Metamodel that:}
   \begin{itemize}
     \item \onslide<2->{can be used to define CouchEdit configurations}
     \item \onslide<3>{abstracts away from implementation details}
   \end{itemize} 
\end{frame}

\begin{frame}
  \frametitle{Related Work}

  

\end{frame}

\begin{frame}
  \frametitle{Design Science Research}

  

\end{frame}



\section{CouchEdit Architecture}

\begin{frame}
  \frametitle{Clear Separation between abstract and concrete Syntax}
\end{frame}


\begin{frame}
  \frametitle{Hypergraph}
\end{frame}

\begin{frame}
  \frametitle{Processors}
  \hspace{-1cm}
  \includegraphics[width=10cm]{images/"component - bus"}
\end{frame}
\chapter{Design Concept}
\label{chap:design}
The proposed artifact effectively is a form of triple graph grammar, meaning it specifies a source model (concrete syntax), a target model (abstract syntax) and a third translation metamodel. The definition of abstract syntax metamodel has already standardized approaches (e.g. Ecore\footnote{\url{https://www.eclipse.org/modeling/emf/}}) and thus is mostly ignored. Furthermore, the proposed design currently only supports a concrete syntax definition composed of graphic primitives, thus complex graphic structures would have to be explored in follow up works. This work focuses on the translation metamodel, that connects concrete and abstract syntax and produces one approach to handling this connection.

The primary goal of the proposed design is to reduce complexity of the CouchEdit framework. To this end, multiple strategies are employed, that either abstract away from CouchEdits implementation details, or introduce ways to streamline the translation from concrete to abstract syntax. Most of these strategies introduce performance overhead, but as this work focuses on the design aspect, performance analysis is also subject to further research. 

\section{Abstraction}
Abstraction important because distances user from implementation details



Graph traversal: 

\begin{itemize}
  \item Graph traversal
  \item utility functions
\end{itemize}



\section{Syntax Processors}
The primary concern when defining a modeling language with clear separation between concrete and abstract syntax, is the question on how to connect these two distinct models. In this point, the designed architecture draws inspiration from Fondement and Baar \cite{fondement_making_2005}. The authors proposed idea of connecting abstract und concrete syntax, using DisplayManagers, serves as a basis that can be built upon. This approach consists of two parts, recognition and synchronization. 

\subsection{Recognition}
The recognition part is concerned with detecting patterns in the concrete syntax that have an abstract syntax representation. For this, fondement and baar proposed the introduction of a further abstraction layer, that composes the graphic primitives and attributes, representing a model element, into display classes. But the Author's keep possible implementation of this abstraction layer open. Furthermore it did not seem necessary to create further abstraction in the CouchEdit architecture \comment{why}. Instead the architecture proposed here, tries a different Approach. For each type of DisplayManager, a set of constraints can be defined on the hypergraph's graphic objects. Whenever a graphic object satisfies all constraints of a given display manager type it is deemed to be a concrete representation of this display manager type and an instances of this display manager class and corresponding model element type are created and connected to the graphic object (fig. \ref{fig:place-recognition}). 

\begin{figure}
  \centering
  \includegraphics[height=8cm]{images/"visualization - place-recognition"}
  \caption{PlaceRecognitionProcessor, adding PlaceDM to a GraphicObject that satisfies constraints}
  \label{fig:place-recognition}
\end{figure}

These constraints can be defined as simple Boolean expressions, that are applied to every graphic object in the graph. The constraints for a Place element, could be defined as follows: 

\begin{lstlisting}[language=OCL]
  self.shape is Circle
  self.allRelatedTo(Contains).isEmpty()
\end{lstlisting} 

These constraints first check if the given GO has the shape of a circle. If that is true, it is also checked if the given GO has any contains relations pointing toward itself. If that is the case, the given circle is contained by another element and thus not clearly identifiable as a Place, as it could also possibly represent a token. A corresponding definition for Transitions could look as following: 

\begin{lstlisting}
  self.shape is Rectangle
\end{lstlisting}

These are barebones requirements to identify concrete syntax representations and they could be extended by any amount of further constraints, to increase the amount of specificity required by the concrete definition.

\subsection{Synchronization}
The Synchronization part is now responsible to make sure, the abstract syntax elements attributes align with the concrete state. This step was explained in detail in \cite{fondement_making_2005}. Fondement and baar propose the usage of OCL Invariants as a mechanism for synchronization. But as the here defined approach does not implement the display class abstraction layer, the concrete representation has to be queried directly. For a Place element there are four aspects that have to be synced:
\begin{enumerate}
  \item Name of the given Place
  \item Number of Tokens this place has
  \item Incoming transitions
  \item Outgoing transitions
\end{enumerate} 

Reliably determining a Places name poses some special challenges and requires further concepts, introduced later. Determining the token number, on the other hand is easily implemented using the given tool. Following the OCL syntax proposed by Fondement and Baar, an invariant syncing this attribute could look something like this:
\begin{lstlisting}[language=OCL]
  context PlaceDM: 
  inv: self.me.tokens = self.go
                            .allRelatedFrom(Contains)
                            .select(go | go.shape is Circle)
                            .count()
\end{lstlisting}

This invariant ensures that the token attribute of the model element is always equal to the number of all GOs with the shape Circle, that are contained by our base graphic object. In a similar fashion, incoming and outgoing Transitions can be defined:
\begin{lstlisting}[language=OCL]
  context PlaceDM: 
  inv:  self.me.incoming = 
            self.go
                .allRelatedTo(ConnectionEnd, 
                    rel | rel.isEndConnection)
                .select(go | go.shape is Line)
                .collect(go | go.relatedFrom(ConnectionEnd))
                .select(go | go.shape is Rectangle)
                .select(go | go.dm <> null)
                .collect(go | go.dm.me.ref())
\end{lstlisting}

This invariant ensures that all Transitions, connected to the given Place, by a line are added to the list of incoming Transitions (fig. \ref{fig:incoming-sync}). This exhibits the usual approach to syncing concrete and abstract syntax. Starting from a DisplayManager, the processors searches along a path of relations and elements, to determine the correct state of a model element.

% What immediately becomes clear is, that the model element side of the invariant is always relatively simple, while querying the concrete syntax and the existing relations can grow in complexity fast. 

\begin{figure}
  \centering
  \includegraphics[height=7.5cm]{images/"visualization - incoming-sync"}
  \caption{A connection line is added to the graph and the PlaceSyncProcessor updates the Place model element}
  \label{fig:incoming-sync}
\end{figure}


\section{Kind System}

When assessing the Incoming Transition Invariant, defined in the last section, it becomes apparent that checking if the connected GO represents a Transition has to be done manually. In the given example, checking if the GO represents a Transitions is no big task as Transition recognition is handled with only one constraint, but when regarding model elements with multiple constraints, this can become a repetitive and inefficient task. 

To Alleviate this problem, the proposed architecture introduces a kind system. A Kind is a sort of meta type, that annotates Elements in the graph, to give further information about their purpose. A Kind is, similar to AttributeBags, a separate Element in the Graph, that is Connected to the Element it annotates, with a relation. It has a single value Attribute, das indicates its purpose. In the same way as the syntax recognition part, the kind system takes a set of constraints for a given kind. Every Element in the graph is then checked against these constraints and if the Element satisfies all of them, the Kind is added to the given Element. This behavior is depicted in fig. \ref{fig:kind-recognition}.

\begin{figure}
  \centering
  \includegraphics[height=7.5cm]{images/"visualization - kind-recognition"}
  \caption{TransitionKindProcessor detects a GO that satisfies constraints and adds KindElement}
  \label{fig:kind-recognition}
\end{figure}

The similarity To the syntax recognition part means that it can actually replace this part, which is done in this Concept. Instead of checking all constraints themselves, syntax recognition processors, only just check if a given GO has a certain Kind and if this is true, the corresponding DisplayManager is added. As shown in figure \ref{fig:Transition-Kind-Recognition}, when a GO, representing a Transition is added, the TransitionKindProcessor, first adds a Kind with the value Transition, the TransitionRecognitionProcessor, then finds the GO that now has a Transition Kind and adds the corresponding Transition DisplayManager.

\begin{figure}[ht]
\centering
\includegraphics[height=11cm]{images/"visualization - Transition-Kind-Recognition"}
\caption{TransitionRecognitionProcessor adds DM after the TransitionKindProcessor has added a Transition Kind}
\label{fig:Transition-Kind-Recognition}
\end{figure}

with the definition of a convenience member function that checks if an Element has a given Kind, the Place incoming transition invariant can now be rewritten as follows: 
\begin{lstlisting}[language=OCL]
  context PlaceDM: 
  inv:  self.me.incoming = 
            self.go
                .allRelatedTo(ConnectionEnd, 
                    rel | rel.isEndConnection)
                .select(go | go.shape is Line)
                .collect(go | go.relatedFrom(ConnectionEnd))
                .select(go | go.hasKind(Transition))
                .collect(go | go.dm.me.ref())
\end{lstlisting}


It is important to note that the Kind system isn't exclusive to GOs that have an abstract representation. For GO's with a corresponding DisplayManager, the type of this DM can just be checked to find out if it connects a model element one is interested in. But the Kind System can be used to recognize all sorts of patterns in the Graph, that would otherwise have to be checked repeatedly.

\section{Plugins}

While the invariant for incoming transitions, is sufficient for an initial example, it has two mayor flaws. Firstly, it is not checking if the connecting line has an arrow end and secondly only lines that are drawn from the Transition towards the Place, are treated as incoming lines, which is defined by the isEndConnection attribute of the ConnectionEnd relation. Fixing these issues in form of an invariant, would be to verbose vor such a common Pattern.

For this reason, a Plugin system is introduced. Plugins are predefined Processors that process specific parts of the Hypergraph. These processing areas are not as integral as the ones solved by the core processors and thus are opt-in. Furthermore certain plugin processors, can be configured which allows them to be suited to certain tasks. One Example of such a plugin would be the TransitionProcessor. It looks for line GraphicObjects and checks if they connect two other GOs. If that's the case, the Processors adds either a TransitionTo relation or a TransitionBetween relation (fig. \ref{fig:transition-plugin}), depending on if the line's ArrowEnds indicate a directed or undirected Transition. Adding this Plugin allows for a final rewrite of the incoming transitions invariant:

\begin{lstlisting}[language=OCL]
  context PlaceDM: 
  inv:  self.me.incoming = 
            self.go
                .allRelatedTo(TransitionTo)
                .select(go | go.hasKind(Transition))
                .collect(go | go.dm.me.ref())
\end{lstlisting}

\begin{figure}
\centering
\includegraphics[height=7.5cm]{images/"visualization - transition-plugin"}
\caption{TransitionProcessors adds TransitionTo relation on detecting a directed line}
\label{fig:transition-plugin}
\end{figure}

\subsection{Label Processor}




\section{Metamodel}


   


\begin{itemize}
  \item abstraction (implemented with extension functions)
  \item OCL like syntax to navigate tree
  \item Graph Transformations 
  \item forcing the user to draw clear graphs
  \item transformation precondition heavy, post condition easy
\end{itemize}
\chapter{Evaluation}
\label{ch:evaluation}
One of the most important parts of a Design Science Research process is the evaluation. The developed artifact has to be analyzed and it has to be established, how well the defined goals are realized by the artifact.

The main criteria that has to be evaluated, is the applicability of the developed artifact. Meaning, it has to be analyzed how well the developed artifact satisfies the defined goal. To this end it seems worthwhile to evaluate the prototypes performance as well as its developer usability. 

\section{Performance}
To provide performance optimization possibilities, \textsc{CouchEdit} was developed with the \texttt{Diff} system in mind. \texttt{Diffs} provide the possibility to calculate changes, without the need for reevaluating the complete hypergraph. On the flipside, this means that all possible states of the graph have to be minded. Using this \texttt{Diff} based approach was evaluated as laborious and error prone, but proved invaluable to improve performance of certain processing tasks \cite{nachreiner_couchedit_2020}. Nachreiner's test results showed that especially language specific processing tasks only took up a small amount of the complete processing time. On basis of this result it was decided, that the developed artifact, primarily concerned with language processing, can produce components that reevaluate the complete graph on every change. Plugin processors, such as the \texttt{LabelProcessor} that is expected to cause high load, are still implemented on application level and thus can make use of the performance benefits granted by the \texttt{Diff} system. 

The actual performance overhead, introduced by this artifact could not be analyzed because of time constraints. Thus, future work would have to evaluate this factor. Furthermore it could be explored if \texttt{Processors}, utilizing the \texttt{Diff} based approach, can be generated from the metamodel, therefore reintroducing the performance benefits. In the current architecture, a \texttt{DM Processors} is triggered every time a change is published, so it can reevaluate the graph. Therefore, a single change done in the frontend will often cause each \texttt{DM Processor} to trigger several times. The metamodel introduces a more structured ordering of \texttt{Processors}. This could be used to split up \texttt{Processors} into sub groups. This way \texttt{Processors} in the same group will calculate changes until no \texttt{Processor} can produce new changes. All changes calculated are then bundled and passed to the next group. This could reduce the number of times a language specific \texttt{Processor}, which reevaluates the complete graph, is triggered.  

\begin{figure}
\centering
\includesvg[width=\linewidth]{images/"component - sub-groups"}
\caption{\texttt{Processors} split into subgroups, where one group is only activated if the previous group has finished processing.}
\label{fig:sub-groups}
\end{figure}

\section{Usability}
Developer usability ask the question on how easy it is to use a tool. The main focus of this research was to develop an artifact that simplifies the process of implementing new modeling syntax configurations for \textsc{CouchEdit}. To this end, it has to be evaluated how well the developed artifact achieves this goal. To find a comprehensive answer to this question, a user study is required. Given the state of the developed artifact as well as the time required to conduct a survey, this goes far beyond the projects scope. Therefore, alternative metrics have to be dissected, in order to determine an inclination regarding this question. 

One metric to highlight is conciseness. As an abstraction of the \textsc{CouchEdit} architecture, the developed artifact should be able to implement similar concepts in less lines of code. The example implementations reflect this. The Petrinets implementation is \petriConfigLoC lines long and produces \petriGeneratedLoC lines of code. Equally, the Statecharts implementation consists of \stateConfigLoC and generates \stateGeneratedLoC lines. In both examples, this means, on average of over 6 lines of source code are generated per line of the configuration. Of course, the code generator most likely generates code that is more verbose than an equivalent written by a developer. therefore this metric is flawed and only serves as a suggestion for the possible usability. 

To further reinforce claims of conciseness, the Statecharts example (\Cref{sec:state-impl}) was modeled as closely as possible after the Statecharts configuration implemented in \cite{nachreiner_couchedit_2020}. An exact replica of the modeling syntax implemented by Nachreiner is not possible. The abstract syntax metamodel defined by them, utilizes \texttt{Relations}, which the AbstractMM defined in \Cref{sec:abstract-syntax} does not support. This results from the close orientation towards Ecore. Furthermore, the \texttt{ConcreteMM} is opinionated in its approach to define configurations and thus, can naturally not provide the same flexibility as an application level implementation has. Nonetheless, the application level implementation is 1900 lines long \cite{nachreiner_couchedit_2020}, and was realized here in \stateConfigLoC lines. This is primarily contributed to the fact that implementations using the metamodel do not have to mind every possible state of the graph, as well as the reduction of a lot of boilerplate code.

While the artifact abstracts away from the implementation details of \texttt{Processors}, the developer still requires certain knowledge about the framework. Most prominently a developer has to understand the hypergraph. Constraints and rules usually require querying of the hypergraph, therefore it is inevitable to know the existing element types as well as possible Relations and their meaning. Furthermore, it has to be understood which changes, different plugins apply to the graph, as well as what functions are available. Nonetheless, the amount of knowledge required, is still reduced.   

\subsection{plugins}
The plugin system proved as a worthwhile addition to the architecture. The example language configurations, described in \Cref{sec:example-configs} show of how complex tasks can be trivialized if the correct plugin is provided. With the plugin system, new \texttt{Processors} can be added to the library of existing processors, should problems arise that are difficult to solve within the generalized metamodel architecture. Plugins are developed on the application level an thus have access to \textsc{CouchEdit}'s complete architecture. On the other hand, this means that developing plugins requires knowledge over the complete \textsc{CouchEdit} framework, which the developed artifact is trying to abstract. Therefore plugins should be implemented as general as possible, so that they can be reused as much as possible. Therein lies the challenge of the plugin system. The usefulness of a plugin depends on how well it can be adapted to different modeling syntaxes. This means that plugins have to be developed with great care. In the example Statecharts implementation (\Cref{sec:state-impl}), the \texttt{LabelProcessor} prototype failed to solve all labeling requirements imposed by the syntax. This demonstrates the volatility of hastily developed plugin. On the other hand, the ConnectionPlugin, thanks to its simplicity was able to provide a concise solution for the given problem.

\section{Discussion of DisplayClasses}
During initial evaluation of the architecture proposed by Fondement and Baar \cite{fondement_making_2005} it was decided to not adapt the concept of \texttt{DisplayClasses}, proposed by the Authors. This decision was made, because their implementation details were not certain. It was estimated, that \texttt{DisplayClasses} would introduce further complexity without providing any significant advantages. with regards to the developed artifact, this turned out as mostly true. Nonetheless, there are certain scenarios as well as considerations towards future work that could make an implementation utilizing \texttt{DisplayClasses} relevant.  

\comment{Something something stricter concrete syntax with (1..1) name for example}


Another scenario that could make \texttt{DisplayClasses} relevant is concerned with syntax feedback in the frontend. \textsc{DiaGen} marks graphical element in a different color, if they are part of a syntactic correct concrete representation \cite{minas_concepts_2002}. Such a feature could be conceivable in future iterations of frontends for \textsc{CouchEdit}. With the pattern system developed in this thesis, there does not exist a direct connection between all \texttt{GOs} and the abstract representation they are part of. Using \texttt{DisplayClasses}, every \texttt{GO} has a direct connection to the abstract representation they are mapped to. Thus, \texttt{DisplayClasses} would trivialize implementation of a feature similar to \textsc{DiaGen}'s.



\section{\comment{Something Something TGG Discussion}}
\comment{Ich sollte wohl irgendwas dazu schreiben, warum ich keinen klassischen TGG approach gewählt hab, aber ich weiß nicht was} 


% \begin{itemize}
%   \item suggestions
%   \item syntactic correctness
%   \item plugin processor needed for specialized tasks
%   \item plugins need to to be well designed (to specialized\\
%         and is general applicability drops,badly designed config model and its a hassle to use )
% \end{itemize}





\section{Perspectives}
\begin{frame}
  \centering
  \hspace{-1cm}
  \huge
  \red{Perspectives}
\end{frame}

\end{document}