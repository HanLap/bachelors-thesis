% !ignore-section
\NeedsTeXFormat{LaTeX2e}


% Font size = 10, DIV = 11 to reduce waste of space %
\documentclass[
a4paper,
% 10pt,
% headsepline,           % Linie zw. Kopfzeile und Text
% doubleside,            % doppelseitig
numbers=noenddot,      % keine Punkte nach den letzten Ziffern in Überschriften
bibliography=totoc,              % LV im IV
%DIV=15,               % Satzspiegel auf 15er Raster, schmalere Ränder   
BCOR=15mm,               % Bindekorrektur
leqno					% equation numbers to the left
% fleqn
%,draft
]{scrbook}
\KOMAoptions{DIV=11} % Neuberechnung Satzspiegel nach Laden von Paket helvet

% be able to include real vector graphics in thesis. You need inkscape installed and your pdflatex command must be called with --shell-escape for this to work.
% TeXstudio: Options → Configure → Commands → pdflatex = pdflatex -synctex=1 -interaction=nonstopmode --shell-escape %.tex
% inkscape=newer to avoid re-export if nothing has changed in the svg. inkscapelatex=false for not using latex to render text (which in turn messes up all text in image). inkscapearea = page so that the SVG size is respected and borders in the SVG are not cut off
%\usepackage[inkscape=newer, inkscapelatex=false, inkscapearea=page]{svg} 


\pagestyle{headings}
\usepackage{blindtext}

% für Texte in deutscher Sprache
\usepackage[ngerman, USenglish]{babel}
\usepackage[utf8]{inputenc}
\usepackage[T1]{fontenc}

% enable HERE positioning %
\usepackage{float}

\usepackage{scrhack}

% Helvetica als Standard-Dokumentschrift
\usepackage[scaled]{helvet}
\renewcommand{\familydefault}{\sfdefault} 


\usepackage{graphicx}
\hbadness=10000

% Literaturverzeichnis mit BibLaTeX // use Biber as Backend; dashed = false to repeat author names
\usepackage[babel]{csquotes}
\usepackage[backend=biber,style=ieee,dashed=false,hyperref,natbib]{biblatex}
\addbibresource{BA.bib}

% Für Tabellen mit fester Gesamtbreite und variabler Spaltenbreite
\usepackage{tabularx} 

% multirow tables
\usepackage{multirow}

% arrows in normal text, not only math
\newcommand*{\textrightarrow}{$ \rightarrow $}

\newcommand*{\textdownarrow}{$ \downarrow $}


% Besondere Schriftauszeichnungen
\usepackage{url}              % \url{http://...} in Schreibmaschinenschrift
\usepackage{color}            % zum Setzen farbigen Textes

\usepackage{setspace}         % Paket für div. Abstände, z.B. ZA
\setlength{\parindent}{0pt}   % kein linker Einzug der ersten Absatzzeile
\setlength{\parskip}{1.4ex plus 0.35ex minus 0.3ex} % Absatzabstand, leicht variabel

% Tiefe, bis zu der Überschriften in das Inhaltsverzeichnis kommen
\setcounter{tocdepth}{2}      % 3 ist Standard
\setcounter{secnumdepth}{3}   % 2 ist Standard

% don't list subsections of appendices in TOC
\usepackage{tocvsec2}


% Mathe
\usepackage{amsfonts}
\usepackage{amssymb}
%\usepackage{amsmath}
% Multilined / aligned command %
\usepackage{mathtools}

% Plots - https://www.overleaf.com/learn/latex/Pgfplots_package
\usepackage{pgfplots}

\pgfkeys{/pgf/number format/.cd,1000 sep={}}
\pgfplotsset{
	compat=newest,
	legend style={at={(0.5,-0.2)},anchor=north}, % legend to the bottom
	legend cell align={left}, % left align legend text
	ymajorgrids=true, 
	grid style=dashed,
	scaled ticks=false, % don't use 10^x notation
	tick label style={/pgf/number format/fixed},
	try min ticks=8,
	tick pos=left % no ticks at the right / top borders
}

% Landscape pages
\usepackage{pdflscape}

% diagonal boxes in table
\usepackage{diagbox}

% resume enumerate https://tex.stackexchange.com/questions/210429/how-can-a-bring-a-middle-paragraph-out-of-the-enumerate-environment
\usepackage{enumitem}

\usepackage[utf8]{inputenc}

% Pseudocode %
\usepackage[lined, algochapter, resetcount, linesnumbered]{algorithm2e}
\SetKwData{Left}{left}\SetKwData{This}{this}\SetKwData{Up}{up}
\SetKwFunction{Union}{Union}\SetKwFunction{FindCompress}{FindCompress}
\SetKwInOut{Input}{input}\SetKwInOut{Output}{output}
\SetKw{Continue}{continue}
\newcommand{\Function}[3]{
	\SetKwBlock{FunctionBlock}{function \textnormal{\textsc{#1} (\emph{#2})}}{end function}
	\FunctionBlock{#3}
}


\usepackage{hyperref,microtype,ifthen} 

\usepackage{xcolor}
% Cleveref so we don't have to write "Section \ref" each time and nameinlink so that the word "Section" also belongs to the PDF link!
\usepackage[nameinlink]{cleveref}
\usepackage{graphicx}


\usepackage{cellspace}
\setlength\cellspacetoplimit{4pt}
\setlength\cellspacebottomlimit{4pt}

% used for subfigures
\usepackage{subcaption}

% used for multi-page graphics
\usepackage{caption}
\usepackage{zref-savepos}
\usepackage{dpfloat}
%\providecommand*{\zsaveposy}{\zsavepos}% support older zref-savepos

% Don't abbreviate figure crefs. Also support algorithm2e listings %

\crefname{figure}{figure}{figures}
\crefname{equation}{formula}{formulas}
\crefname{algorithm}{algorithm listing}{algorithm listings}

\creflabelformat{equation}{#2#1#3}

% Graphen und sonstige Zeichnungen
\usepackage{tikz}
\usetikzlibrary{shapes.geometric}
\usetikzlibrary{shapes.misc}
\usetikzlibrary{positioning}
\usetikzlibrary{calc}

% Layout
\usepackage[scale=0.70, marginratio={4:5, 3:4}, ignoreall, headsep=8mm]{geometry}
\setlength{\parskip}{1.4ex plus 0.35ex minus 0.3ex}
\renewcommand\arraystretch{1.3} % höhere Zeilen in Tabellen
\clubpenalty10000  % keine Schusterjungen
\widowpenalty10000 % keine Hurenkinder
\setcounter{tocdepth}{3} % Tiefe, bis zu der Überschriften in das Inhaltsverzeichnis kommen


% vertical table headers https://tex.stackexchange.com/questions/98388/how-to-make-table-with-rotated-table-headers-in-latex
\usepackage{adjustbox}
\usepackage{array}
\usepackage{booktabs}

% partial function arrow https://tex.stackexchange.com/questions/47142/how-to-tex-an-arrow-with-vertical-stroke %
\newcommand\pto{\mathrel{\ooalign{\hfil$\mapstochar\mkern5mu$\hfil\cr$\to$\cr}}}

% autorefs %
\def\sectionautorefname{section}

% Beispiele für Quellcode
\usepackage{listings}

\lstdefinelanguage{Kotlin}{
  comment=[l]{//},
  commentstyle={\color{gray}\ttfamily},
  emph={delegate, filter, first, firstOrNull, forEach, lazy, map, mapNotNull, println, return@},
  emphstyle={\color{OrangeRed}},
  identifierstyle=\color{black},
  keywords={abstract, actual, as, as?, break, by, class, companion, continue, data, do, dynamic, else, enum, expect, false, final, for, fun, get, if, import, in, interface, internal, is, null, object, override, package, private, public, return, set, super, suspend, this, throw, true, try, typealias, val, var, vararg, when, where, while},
  keywordstyle={\color{NavyBlue}\bfseries},
  morecomment=[s]{/*}{*/},
  morestring=[b]",
  morestring=[s]{"""*}{*"""},
  ndkeywords={@Deprecated, @JvmField, @JvmName, @JvmOverloads, @JvmStatic, @JvmSynthetic, Array, Byte, Double, Float, Int, Integer, Iterable, Long, Runnable, Short, String},
  ndkeywordstyle={\color{BurntOrange}\bfseries},
  sensitive=true,
  stringstyle={\color{ForestGreen}\ttfamily},
}



\lstset{language=Kotlin,
  showstringspaces=false,
  frame=single,
  numbers=left,
  basicstyle=\ttfamily,
  numberstyle=\tiny
  captionpos=b,
  numbers=left,
  basicstyle=\singlespacing\ttfamily,
  numberstyle=\smaller\ttfamily,
  tabsize=4
}

\makeatletter
\AtBeginDocument{\@ifpackageloaded{amsmath}{\@mathmargin\z@}{}}%
\makeatother

% hier Namen etc. einsetzen
\newcommand{\fullname}{Florian Lappe}
\newcommand{\email}{florian.lappe@uni-ulm.de}
\newcommand{\titel}{}
\newcommand{\untertitel}{Development and Evaluation of a Metamodel to Define Modeling Syntaxes for CouchEdit}
\newcommand{\jahr}{2020}
\newcommand{\abgabedatum}{August 2020}
%\newcommand{\abschlussarbeit}{Bachelorarbeit}
\newcommand{\abschlussarbeit}{Bachelorarbeit}
\newcommand{\matrikelnummer}{922114}
\newcommand{\gutachterA}{Prof.\ Dr.\ Matthias\ Tichy}
\newcommand{\gutachterB}{}
\newcommand{\betreuer}{Dr.\ Alexander\ Raschke}

% hier die Fakultät auswählen
%\newcommand{\fakultaet}{---  Im Quellcode anpassen nicht vergessen! ---}
\newcommand{\fakultaet}{Ingenieurwissenschaften, Informatik und\\Psychologie}
%\newcommand{\fakultaet}{Mathematik und\\Wirtschafts-\\wissenschaften}
%\newcommand{\fakultaet}{Medizin}
%\newcommand{\fakultaet}{Naturwissenschaften}

% hier das Institut einsetzen
\newcommand{\institut}{Institut für Softwaretechnik und Programmiersprachen}


\newcommand{\comment}[1]{\textcolor{red}{#1}}

% Informationen, die LaTeX in die PDF-Datei schreibt
\pdfinfo{
  /Author (\fullname)
  /Title (\titel)
  /Producer     (pdfeTex 3.14159-1.30.6-2.2)
  /Keywords ()
}

\selectlanguage{ngerman}

\usepackage{hyperref}
\hypersetup{
pdftitle=\titel,
pdfauthor=\fullname,
pdfsubject={metamodeling, metamodel},
colorlinks=false,
pdfborder=0 0 0	% keine Box um die Links!
}

% Trennungsregeln
\hyphenation{Sil-ben-trenn-ung} 

\begin{document}
\frontmatter % ab hier römische Seitenzahlen


% Titelseite
\newgeometry{left=1.9cm, right=1.9cm, top=2.9cm, bottom=2.8cm}
\begin{titlepage}
  \fontfamily{phv}\selectfont % Helvetica als Schriftart
	\hfill\includegraphics[height=2.0cm]{images/logo_100_sRGB}\\[3.5cm] % Uni Ulm Logo 
	\begin{flushright}
		\Huge \textbf{\titel}\\[0.2cm]
		\fontsize{19}{20}\selectfont \textbf{\untertitel}\\
	\end{flushright}
	
	\vfill\hfill
	\parbox[t]{4.6cm}{
		\singlespacing
		\large
		\textbf{\fullname}\\
		\\
		Universität Ulm\\
		\\
		Fakultät für\\
		Ingenieurwissenschaften\\
		und Informatik\\
		\\
		Institut für\\
		Programmiermethodik\\
		und Compilerbau\\
		\\
		\abgabedatum\\
		\\
		{\abschlussarbeit} im\\
		Studiengang Informatik
	}
\end{titlepage}
\restoregeometry


% Abstract
\clearpage
\thispagestyle{empty}
\chapter*{Abstract}

% !/ignore-section

Abstract Abstract Abstract Abstract Abstract Abstract Abstract Abstract Abstract,
Abstract Abstract Abstract Abstract Abstract Abstract Abstract Abstract Abstract.
Abstract Abstract Abstract Abstract Abstract Abstract Abstract Abstract Abstract,
Abstract Abstract Abstract Abstract Abstract Abstract Abstract Abstract Abstract.

Abstract Abstract Abstract Abstract Abstract Abstract Abstract Abstract Abstract,
Abstract Abstract Abstract Abstract Abstract Abstract Abstract Abstract Abstract.
Abstract Abstract Abstract Abstract Abstract Abstract Abstract Abstract Abstract,
Abstract Abstract Abstract Abstract Abstract Abstract Abstract Abstract Abstract.

% !ignore-section
{
	\null
	\small
	\vfill
	\begin{center}
		\begin{tabular}{l l}
			Erstgutachter:  & \gutachterA \\
			Zweitgutachter: & \gutachterB \\
			Betreuer:       & \betreuer \\
		\end{tabular}\\[1cm]
		Fassung \today\\
		  \copyright~\jahr~\fullname\\[0.5em]
		% Wenn Sie Ihre Arbeit unter einer freien Lizenz bereitstellen möchten, können Sie die nächste Zeile in Ihren Code aufnehmen. Bitte beachten Sie, dass Sie hierfür an allen Inhalten, inklusive enthaltener Abbildungen, die notwendigen Rechte benötigen! Beim Veröffentlichungsexemplar Ihrer Dissertation achten Sie bitte darauf, dass der Lizenztext nicht den Angaben in den Metadaten der genutzten Publikationsplattform widerspricht. Nähere Information zu den Creative Commons Lizenzen erhalten Sie hier: https://creativecommons.org/licenses/
		This work is licensed under the Creative Commons Attribution 4.0 International (CC BY 4.0) License. To view a copy of this license, visit \href{https://creativecommons.org/licenses/by/4.0/}{https://creativecommons.org/licenses/by/4.0/} or send a letter to Creative Commons, 543 Howard Street, 5th Floor, San Francisco, California, 94105, USA. \\
		
		Satz: PDF-\LaTeXe
	\end{center}
}


% Inhaltsverzeichnis
\tableofcontents

\mainmatter % ab hier wieder normale Seitenzahlen






% Inhalt (am besten mit \input in extra Dateien auslagern)
\chapter{Introduction}


Modeling languages have long played an important role in software engineering. Well designed models can abstract complex systems and provide visual aid in understanding them. Furthermore in form of the Business Process Model Notation (BPMN) they are used to define and automate processes. Today, research in the area of Model Driven Engineering (MDE), a paradigm centered around models, with the intent to generate code bases and whole systems from them, the importance of models is rising even more.

As these tasks require syntactic correctness of used models, modeling tools become an essential part of an engineer's workflow. Especially visual modeling tools provide in theory, an intuitive and user friendly way to design models. But current graphical modeling tools tend to constraint users in unintuitive ways and deliver sub par user experience (UX). this usually arises from a tight coupling between a modeling tools user interface (UI) and the underlying model. As the model's syntax is usually inflexible, the UI has to make restrictions to adhere to this syntax. this often creates problems for the user, for example connections can only be drawn between two existing states, or deleting a node will result in all its children being deleted as well.

To amend these usability woes, L. Nachreiner proposed a novel modeling framework, called CouchEdit \cite{nachreiner_couchedit_2020}. This framework decouples user interface and model syntax by introducing different models for both. Instead of relying directly on the syntax of the model that is being designed, in the CouchEdit architecture the user interface is using a render model that only consists of nodes that are rendered in the Modeling tool, called concrete syntax. On the other hand, the actual models syntax now stands on its own, called abstract syntax. To translate between concrete and abstract syntax, a syntax metamodel is utilized. CouchEdit at its core was designed to be general purpose, meaning it can be rewritten to adhere to any model syntax. But to realize this in the current implementation, the source code has to be changed directly, which is error prone, convoluted and requires some understanding of CouchEdits internal architecture.

To create a more developer friendly and flexible way of adapting CouchEdit to different modeling syntaxes, this design research aims to propose a new metamodel, that can be used to create modeling syntax definitions which are usable by CouchEdit. Furthermore a conceptual parser is to be developed and implemented, that provides proof of concept on how this newly developed metamodel interacts with the CouchEdit architecture.
\chapter{Related Work}
\label{sec:related_work}

\comment{intro}


\section{Making Metamodels Aware of Concrete Syntax}
\label{sec:fondement}
in their work, Making Metamodels Aware of Concrete Syntax \cite{fondement_making_2005}, F. Fondement and T. Baar argue that, while abstract syntax definitions are standardized, most language specifications keep the concrete syntax informal. To solve this problem, they propose an approach to defining the concrete syntax and how to link it to the abstract representation.

For this, the authors complement every class of the abstract syntax with a corresponding display scheme. This display scheme is compose of two parts, an iconic and a constraining part. The iconic part defines a set of DisplayClasses, these DisplayClasses group Graphical Objects together into visual representation object. On the other hand, the constraining part links these DisplayClasses to an abstract syntax element. This link is realized using DisplayManagers. A DisplayManager serves as connection between exactly one model element of the abstract syntax and one display object and has the task of syncing the abstract to the concrete representation. An example of this architecture is depicted in figure \ref{fig:fondement_dm}, for a petri net place element. The graphical primitives, a place is composed of, are mapped to a place DisplayClass, to build a places iconic part. This iconic representation is then attached to a place model element, using a place DisplayManager.


\begin{figure}[H]
  \centering
  \includegraphics[width=\linewidth]{images/"csd - fondement-example"}
  \caption{Example representation of a petri net place element}
  \label{fig:fondement_dm}
\end{figure}

A DisplayManager has to keep abstract and concrete representation in sync, for this F. Fondement and T. Baar utilize OCL invariants, which are defined on the DisplayerManager. For example, an invariant to sync the name of place display schemes could look as following:

\begin{lstlisting}[language=OCL]
  context PlaceDM
  inv: self.me.name = self.vo.name.text
\end{lstlisting}

\comment{kinda hard cut}

The Authors keep open, how the mapping from graphical primitives to a display object could be implemented. while the metamodel proposed in this thesis does not introduce an extra layer of abstraction in form of these DisplayClasses, it still is inspired heavily by F. Fondement's and T. Baar's work. Especially the constraining part that utilizes DisplayManagers to sync abstract and concrete syntax, served as primary inspiration for the proposed architecture.


\comment{we need some more related work.}
\chapter{Fundamentals}
\label{chap:fundamentals}

this chapter lays out the knowledge foundation, required in later sections \comment{(*barf* horribly written)}. First an outline of the applied research process is given. following that, a comprehensive view of the CouchEdit architecture is given. This will be needed later on to understand certain design decisions. Thereafter, a foundation of modeling languages is laid out, as well as some modeling syntaxes. Finally, a summary of related work is and their impact on this thesis will be presented. 

\section{Methodology}
As this research strived to develop a new metamodel suitable for the CouchEdit framework, it was conducted in accordance to the Design Science Research (DSR) approach. According to V. Vaishnavi et al. a design science research process consists of five steps \cite{vaishnavi_design_2004}.

\subsection{Problem Awareness}
The first step of a design science research is the identification of existing problems. As specified in section \ref{sec:problem_statement}, it was identified that the current CouchEdit implementation lacks a developer friendly way to configure it for different modeling syntaxes and that it is unclear how a metamodel for this architecture would look like.

\subsection{Suggestion}
With a clear definition of the problem, objectives can be proposed which have to be achieved in order to solve this problem. The first objective of this research is to develop a metamodel for the CouchEdit architecture. To be more precise, a metamodel is to be designed, that can be used to specify model syntax definitions which map concrete graphical syntaxes to Abstract syntax models and is applicable to CouchEdit's architecture. The second objective is to implement a prototype that provides proof of concept for the applicability of the design metamodel.


\subsection{Development}
The primary goal of a design science research is the development of artifacts.
The first artifact to be developed in this research will be a metamodel, that can be used to define new modeling syntaxes for a relaxed conformance editor and is applicable to CouchEdit's architecture. The second artifact that is to be developed, is a prototypical code generator that can translate the developed metamodel into a CouchEdit implementation which will be able to process the defined modeling syntax. Furthermore it was decided to implement a simple DSL that eases the process of defining an instance of the given metamodel.

To this end, the first sub step of the development stage is a comprehensive analysis of CouchEdit's architecture. In his work L. Nachreiner describes in detail, which modeling features the framework covers and how they are implemented \cite{nachreiner_couchedit_2020}. The goal is to develop an approach that can streamline the process of defining modeling syntaxes while adhering to CouchEdit's specification. 

\comment{ich hab mich echt kein bischen an meinen definierten prozess gehalten. todo}

For each of these features, a sub metamodel has to be designed that can be used to define the given feature (RQ1). The approaches of \cite{minas_specifying_2001} and \cite{fondement_making_2005} provide a DSL implementation from which concepts for potential metamodels could be derived. If for one of the features, a suitable metamodel cannot be found, it would have to be evaluated if the feature could be simplified or even dropped (R1.1). After designing sub metamodels for all features, they then have to be composed together, which results in a first version of an applicable metamodel. 

in the next sub step, the code generator will be implemented, this is an iterative step. first, an implementation will be developed on the basis of the designed metamodel, this should reveal further requirements imposed by CouchEdits architecture (RQ2). The metamodel then has to be adjusted to account for these requirements, which in turn requires changes of the implementation until no further requirements can be deduced. Because of time constraints it will not be possible to implement all of the metamodel's features, instead the prototype will cover a minimal feature set that still suffices to demonstrate the metamodel's applicability for selected modeling syntaxes.

\subsection{Evaluation}
Now that the desired artifacts are fully developed it has to be evaluated how well they do their designated task. The metamodel as the primary artifact of this research has to be evaluated in terms of its applicability to its designated domain. To this end, the metamodel will be demonstrated on the basis of different modeling syntaxes, this should highlight areas in which the metamodel's design excels, as well as design flaws and limitations (RQ3). If time allows, potential flaws can be addressed by returning back to the development phase and revising the artifacts, otherwise flaws are to be highlighted so that future works can address them.

\subsection{Conclusion}
The conclusion stage marks the end of a DSR and the results are written up. This works written part is composed of this thesis as well as all written source code and documentation.


\section{CouchEdit}
\label{sec:CouchEdit}
As noted in section \ref{sec:problem_statement}, the CouchEdit framework is based on the ideas of Van Tendeloo, et al. \cite{van_tendeloo_concrete_2017}. In their paper, the authors criticize that classic approaches to modeling frontends usually create one monolithic construct, which handles both the graphical model display as well as the corresponding abstract syntax. This creates tight coupling between the two, which causes low flexibility of the UI and high effort requirements, to implement new features. 

They propose to solve this, by defining two metamodels, one describing the graphical syntax and one the abstract syntax. The render syntax metamodel specifies graphic primitives that a given graphical modeling syntax is made of. instances of this metamodel are managed by the frontend and a user can manipulate this concrete representation directly by adding, deleting or modifying Elements in the UI (if the frontend allows changes). On The other side stands the abstract syntax metamodel, it describes the conceptual representation of a modeling syntax. Instances of this metamodel are managed by the backend. To then connect these two metamodels, a further metamodel, called concrete syntax metamodel, can be defined. This concrete syntax metamodel specifies a connection that maps the concrete graphical representation onto an abstract representation (Figure \ref{fig:transmm}). 

\begin{figure}
  \centering
  \includegraphics[width=.7\linewidth]{images/"presentation - transfere-metamodel"}
  \caption{Depiction of CouchEdit's approach to separating graphical and abstract representation}
  \label{fig:transmm}
  \end{figure}

Following this architecture, CouchEdit separates frontend and abstract syntax model. The frontend at its core is a simple vector drawing application that implements further functionality for diagrams. A Backend then has the responsibility to analyze these vector graphics and map them onto a corresponding abstract syntax representation. For this, the backend utilizes a set of independent components, that each are responsible for detecting different aspects of the diagram. Each Component contains its own state. When a component makes changes to its state, it also publishes these changes so that other components, interested in these changes, can use them to calculated follow up changes.  

This architecture, has the advantage of being very detangled. this allows for components to be improved in isolation from the rest of the system and the clear separation of concerns makes it easier to understand, which component is responsible for what task. Furthermore, components can be swapped out depending on what tasks are needed for the configured modeling syntax.



\subsection{Data Model}
CouchEdit uses a hypergraph of Elements and Relations to represent its data. Each component has its own instance of this graph that only contains the area of the graph, that is interesting to this component. A component can manipulate its personal graph and make these changes known to the system, so that other components can use them as well.


\subsubsection{Elements}
\texttt{Elements} are the base type of CouchEdit's hypergraph. Every type of node in this graph inherits from the \texttt{Element} type. As all hypergraph instances should be independent from each other, it is not possible to access \texttt{Elements} via their object reference. Thus every \texttt{Element} has an unique id that kann be used to reliably identify it across multiple graphs. \texttt{Elements} also have a \texttt{probability} attribute that denotes how probable it is that this \texttt{Element} is the result of a correct interpretation of the given hypergraph. This \texttt{probability} object can be set to \texttt{Explicit}, which means that, either this \texttt{Element} describes a factual truth or it was marked by the user as not to be changed. Furthermore,\texttt{Element} subtypes can introduce further attributes that contain important information for the given type.

\subsubsection{Relations}
\texttt{Relations} represent edges of the hypergraph. Each \texttt{Relation} connects a set of source \texttt{Elements} to a set of target \texttt{Elements}. They also inherit from the \texttt{Element} type and thus can recursively be connected with relations themselves. The connected vertices are referenced using the \texttt{ElementReference} class. This class holds an \texttt{Elements} unique id as well as its type, which then can be used to find a referenced \texttt{Element} in the hypergraph. Figure \ref{fig:relations} shows the implementation of \texttt{Relations}. 

\begin{figure}[ht]
  \centering
  \includegraphics[width=.8\linewidth]{images/"csd - relation"}
  \caption{Class diagram of Relations}
  \label{fig:relations}
\end{figure}

 While \texttt{Relations} can connect multiple source and target vertices, the most common \texttt{Relation} type is the \texttt{OneToOneRelation}, it describes a connection between exactly one source and one target \texttt{Element}. A common example, for such a \texttt{OneToOneRelation}, is the \texttt{Contains} relations, which defines that one \texttt{Element} encloses another.  


\subsubsection{Graphic Objects}
\comment{this chapter still is a mess}
graphic objects (GO) represent atomic elements, that the user interacts with, via the Frontend. An important decision, when designing an Editor is the question on how granular singular graphic objects should be. 

On a basic level, GOs can consist of either pixels or vectors. a pixel based Approach would require a visual recognition preprocessor, that composes the state of pixels in to comprehensible graphic primitives. While such form of image recognition is outside the scope of the CouchEdit project, similar projects such as FlexiSketch \cite{wuest_flexisketch_2015} show the application of this approach.

Most modeling approaches with some form of abstract syntax processing, build graphic elements with the purpose of representing a specific element of the abstract syntax. As a result these graphic elements are very strict in their graphical representation. 

CouchEdit Follows the Classification of Costagliola, et al. \cite{costagliola_classification_2002}. This classification defines a graphic element on its lowest level as a primitive graphic shape (e.g circle, line) \comment{(read up if thats actually what costagliola are doing)}. Via sub-classing, a single graphic object can increase in complexity. Such complex graphic object can have  


In his work, The “Physics” of Notations: Toward a Scientific Basis for Constructing Visual Notations in Software Engineering \cite{moody_physics_2009}, Moody specifies eight variables that can be applied to a single graphic primitive. These attributes are separated into factors that have direct influence on the relations between GOs and   Position
(horizontal and vertical), shape, size, orientation, color, brightness and texture. While the 



\begin{itemize}
  \item attribute bag
\end{itemize}

\subsubsection{HotSpotDefinitions}


\subsection{Application Structure}
CouchEdit builds around independent components, called \texttt{Processors} that can be enabled or disabled as needed in a given use case. To this end, the center piece of CouchEdit's architecture is the \texttt{ModificationBusManager}. Every \texttt{Processor} (including the frontend) can be connected to this bus manager (Figure \ref{fig:processors}). if one \texttt{Processor} publishes a change of the hypergraph, the \texttt{ModificationBusManager} propagates this change to all \texttt{Processors} connected. Each \texttt{Processor} can specify which \texttt{Element} types it is interested and will only receive updates for \texttt{Elements} that match one of those types. 

% \begin{figure}
% \centering
% \includegraphics[width=.7\linewidth]{images/"component - bus"}
% \caption{Component diagram describing the underlying \texttt{ModificationBusManager} that connects all processors in the system.}
% \label{fig:bus}
% \end{figure}

\subsubsection{Core Processors}
CouchEdit implements a set of \texttt{Processors} that are integral to most modeling syntaxes. These so called core \texttt{Processors} currently are: 
\begin{description}
  \item[SpatialAbstractor] This \texttt{Processor} has the responsibility to calculate how \texttt{GOs} are positioned to each other. Possible position \texttt{Relations} include: \texttt{RightOf}, \texttt{BottomOf}, \texttt{Intersect}, etc..
  \item[ConnectionEndDetector] The \texttt{ConnectionEndDetector} finds line \texttt{GOs} that could represent a connection to another \texttt{GO}. it Then adds \texttt{ConnectionEnds} relations from the line to the other \texttt{GO}. Furthermore, the \texttt{ConnectionEnd} relation has the attribute \texttt{isEndConnection} that defines, if this relation origins from the ending point of the line.
  \item[Containment] The \texttt{Containment processor}, checks if one \texttt{GO} is contained by another. Meaning, if one \texttt{GO} completely encompasses another \texttt{GO}, the \texttt{Processor} adds a \texttt{Contains} relation from the surrounding \texttt{GO} to the contained one.
\end{description}

\subsection{Services}
\label{sec:services}
\texttt{Processors} may sometimes require to gather additional information, such as. For example it is often needed to find all \texttt{Elements} that are connected to a given \texttt{Element} by a certain \texttt{Relation} type.
To this End, CouchEdit provides \texttt{Services}. \texttt{Services} are separate objects that provide a set of functions. A \texttt{Service} can be required by multiple different \texttt{Processors} and thus must be effectively stateless.



\subsection{Suggestions}
\comment{todo}

\begin{itemize}
  \item compartmentHotSpotDefinition
\end{itemize}

\section{Modeling Languages}
\label{sec:modeling-languages}
This thesis discusses modeling syntaxes and how they can be handled in the CouchEdit architecture. The following chapters will demonstrate the developed artifact with the help of modeling syntax examples. thus the two syntaxes that will be used in this thesis are defined here.

% \subsection{Graphical Language Theory}
\subsection{Petrinets Syntax}
\label{sec:petrinets}
The first modeling syntax introduced are Petrinets. Petrinets are easy to understand and provide simple abstract and concrete syntax. Thus they serve as a suitable tool for explaining concepts in the following chapters. Figure \ref{fig:petrinets_metamodel} shows the abstract syntax metamodel of the used Petrinets syntax. The Petrinets conceptual representation consists of places and transitions. Places also have a token count. each place can have a variable amount of incoming and outgoing transitions, while transitions can have any amount of incoming and outgoing places. 

\begin{figure}[H]
  \centering
  \includegraphics[width=.7\linewidth]{images/"csd - petrinet-metamodel"}
  \caption{Metamodel for a simple Petrinet abstract syntax}
  \label{fig:petrinets_metamodel}
\end{figure}

The graphic primitives, a concrete representation of Petrinets is composed of are listed in table \ref{tab:petri-primitives}. places are usually represented as circles, while transitions are depicted as slender rectangles. The number of small black circles inside a place represent this places token count. Furthermore places and transitions possess a label close to their bounding box, that determines their name. Lastly places and transitions have directed connection lines between them. Each connection points from a place or transition towards an element of the opposite type and represents an outgoing connection for the source element and an incoming connection for the target element. A simple concrete instantiation and its corresponding abstract representation is shown in figure \ref{fig:petrinets_example}.

\begin{table}[ht]
  \centering
\begin{tabular}[width=.1\linewidth]{| Sc | Sc | Sc | Sc | Sc |}
  \hline
  Place & Transition & Token & Label & Connection 
  \\
  \hline
  \includegraphics[width=.1\linewidth]{images/"petrinet - place"} 
  & 
  \includegraphics[width=.1\linewidth]{images/"petrinet - transition"} 
  & 
  \includegraphics[width=.1\linewidth]{images/"petrinet - token"}
  & 
  \includegraphics[width=.1\linewidth]{images/"petrinet - label"}
  & 
  \includegraphics[width=.1\linewidth]{images/"petrinet - connection"} 
  \\
  \hline
\end{tabular}
\caption{graphic primitives used to describe Petrinets}
\label{tab:petri-primitives}
\end{table}

\begin{figure}[ht!]
  \centering
  \begin{subfigure}[t]{.4\textwidth}
    \centering
    \includegraphics[width=.9\linewidth]{images/"petrinet - example"}
    \caption{concrete syntax}
    \label{subfig:petriconcrete}    
  \end{subfigure}
  \begin{subfigure}[t]{.45\textwidth}
    \centering
    \includegraphics[width=\linewidth]{images/"csd - petrinet-example"}
    \caption{abstract syntax}
    \label{subfig:petriabstract}    
  \end{subfigure}
  \caption{concrete and abstract representation for a simple Petrinet example}
  \label{fig:petrinets_example}
\end{figure} 

\subsection{Statechart Syntax}
\label{sec:statecharts}
as a second example, Statecharts were chosen. Statecharts describe a state machine with the characteristic, that each state kann define its own sub state machines. Figure \ref{fig:statechartmm} describes the abstract syntax metamodel that is used. This does not describe the complete Statecharts syntax, rather a sub set which can be used to highlight certain areas of the implementation. Statecharts define a more complex syntax and thus will serve as an example for the applicability of the developed artifact.

\begin{figure}
\centering
\includegraphics[width=.7\linewidth]{images/"csd - new-statechart-metamodel"}
\caption{Abstract Metamodel of Statecharts}
\label{fig:statechartmm}
\end{figure}

Statecharts are primarily composed of \texttt{stateElements} and \texttt{Transitions}.

\texttt{StateElements} are split into \texttt{States} and \texttt{PseudoStates}. \texttt{PseudoStates} have one of several kinds, defined by the \texttt{PseudoStateKind}. Possible types are initial, final and choice


\texttt{States} are called simple \texttt{State} if they posses no sub state machine. In the case they do encompass a sub state machine they are called compound \texttt{State}. A compound \texttt{State} can have multiple sub states, in this case they can be split apart using a dashed separator line, to create \texttt{Regions}. Each \texttt{Region} contains its own state machine.


\begin{itemize}
  \item talk about label positioning
  \item orthogonal regions
\end{itemize}


\begin{figure}
\centering
\includegraphics[width=.7\linewidth]{images/"visualization - statechart-example"}
\caption{Concrete representation of an example Statechart}
\label{fig:statechart-example}
\end{figure}

concepts ignored:
\begin{itemize}
  \item entry/ exit actions
  \item event parsing (guards)
\end{itemize}

\chapter{Design Concept}
\label{chap:design}
The proposed artifact effectively is a form of triple graph grammar, meaning it specifies a source model (concrete syntax), a target model (abstract syntax) and a third translation metamodel. The definition of abstract syntax metamodel has already standardized approaches (e.g. Ecore\footnote{\url{https://www.eclipse.org/modeling/emf/}}) and thus is mostly ignored. Furthermore, the proposed design currently only supports a concrete syntax definition composed of graphic primitives, thus complex graphic structures would have to be explored in follow up works. This work focuses on the translation metamodel, that connects concrete and abstract syntax and produces one approach to handling this connection.

The primary goal of the proposed design is to reduce complexity of the CouchEdit framework. To this end, multiple strategies are employed, that either abstract away from CouchEdits implementation details, or introduce ways to streamline the translation from concrete to abstract syntax. Most of these strategies introduce performance overhead, but as this work focuses on the design aspect, performance analysis is also subject to further research. 

\section{Abstraction}
Abstraction important because distances user from implementation details



Graph traversal: 

\begin{itemize}
  \item Graph traversal
  \item utility functions
\end{itemize}



\section{Syntax Processors}
The primary concern when defining a modeling language with clear separation between concrete and abstract syntax, is the question on how to connect these two distinct models. In this point, the designed architecture draws inspiration from Fondement and Baar \cite{fondement_making_2005}. The authors proposed idea of connecting abstract und concrete syntax, using DisplayManagers, serves as a basis that can be built upon. This approach consists of two parts, recognition and synchronization. 

\subsection{Recognition}
The recognition part is concerned with detecting patterns in the concrete syntax that have an abstract syntax representation. For this, fondement and baar proposed the introduction of a further abstraction layer, that composes the graphic primitives and attributes, representing a model element, into display classes. But the Author's keep possible implementation of this abstraction layer open. Furthermore it did not seem necessary to create further abstraction in the CouchEdit architecture \comment{why}. Instead the architecture proposed here, tries a different Approach. For each type of DisplayManager, a set of constraints can be defined on the hypergraph's graphic objects. Whenever a graphic object satisfies all constraints of a given display manager type it is deemed to be a concrete representation of this display manager type and an instances of this display manager class and corresponding model element type are created and connected to the graphic object (fig. \ref{fig:place-recognition}). 

\begin{figure}
  \centering
  \includegraphics[height=8cm]{images/"visualization - place-recognition"}
  \caption{PlaceRecognitionProcessor, adding PlaceDM to a GraphicObject that satisfies constraints}
  \label{fig:place-recognition}
\end{figure}

These constraints can be defined as simple Boolean expressions, that are applied to every graphic object in the graph. The constraints for a Place element, could be defined as follows: 

\begin{lstlisting}[language=OCL]
  self.shape is Circle
  self.allRelatedTo(Contains).isEmpty()
\end{lstlisting} 

These constraints first check if the given GO has the shape of a circle. If that is true, it is also checked if the given GO has any contains relations pointing toward itself. If that is the case, the given circle is contained by another element and thus not clearly identifiable as a Place, as it could also possibly represent a token. A corresponding definition for Transitions could look as following: 

\begin{lstlisting}
  self.shape is Rectangle
\end{lstlisting}

These are barebones requirements to identify concrete syntax representations and they could be extended by any amount of further constraints, to increase the amount of specificity required by the concrete definition.

\subsection{Synchronization}
The Synchronization part is now responsible to make sure, the abstract syntax elements attributes align with the concrete state. This step was explained in detail in \cite{fondement_making_2005}. Fondement and baar propose the usage of OCL Invariants as a mechanism for synchronization. But as the here defined approach does not implement the display class abstraction layer, the concrete representation has to be queried directly. For a Place element there are four aspects that have to be synced:
\begin{enumerate}
  \item Name of the given Place
  \item Number of Tokens this place has
  \item Incoming transitions
  \item Outgoing transitions
\end{enumerate} 

Reliably determining a Places name poses some special challenges and requires further concepts, introduced later. Determining the token number, on the other hand is easily implemented using the given tool. Following the OCL syntax proposed by Fondement and Baar, an invariant syncing this attribute could look something like this:
\begin{lstlisting}[language=OCL]
  context PlaceDM: 
  inv: self.me.tokens = self.go
                            .allRelatedFrom(Contains)
                            .select(go | go.shape is Circle)
                            .count()
\end{lstlisting}

This invariant ensures that the token attribute of the model element is always equal to the number of all GOs with the shape Circle, that are contained by our base graphic object. In a similar fashion, incoming and outgoing Transitions can be defined:
\begin{lstlisting}[language=OCL]
  context PlaceDM: 
  inv:  self.me.incoming = 
            self.go
                .allRelatedTo(ConnectionEnd, 
                    rel | rel.isEndConnection)
                .select(go | go.shape is Line)
                .collect(go | go.relatedFrom(ConnectionEnd))
                .select(go | go.shape is Rectangle)
                .select(go | go.dm <> null)
                .collect(go | go.dm.me.ref())
\end{lstlisting}

This invariant ensures that all Transitions, connected to the given Place, by a line are added to the list of incoming Transitions (fig. \ref{fig:incoming-sync}). This exhibits the usual approach to syncing concrete and abstract syntax. Starting from a DisplayManager, the processors searches along a path of relations and elements, to determine the correct state of a model element.

% What immediately becomes clear is, that the model element side of the invariant is always relatively simple, while querying the concrete syntax and the existing relations can grow in complexity fast. 

\begin{figure}
  \centering
  \includegraphics[height=7.5cm]{images/"visualization - incoming-sync"}
  \caption{A connection line is added to the graph and the PlaceSyncProcessor updates the Place model element}
  \label{fig:incoming-sync}
\end{figure}


\section{Kind System}

When assessing the Incoming Transition Invariant, defined in the last section, it becomes apparent that checking if the connected GO represents a Transition has to be done manually. In the given example, checking if the GO represents a Transitions is no big task as Transition recognition is handled with only one constraint, but when regarding model elements with multiple constraints, this can become a repetitive and inefficient task. 

To Alleviate this problem, the proposed architecture introduces a kind system. A Kind is a sort of meta type, that annotates Elements in the graph, to give further information about their purpose. A Kind is, similar to AttributeBags, a separate Element in the Graph, that is Connected to the Element it annotates, with a relation. It has a single value Attribute, das indicates its purpose. In the same way as the syntax recognition part, the kind system takes a set of constraints for a given kind. Every Element in the graph is then checked against these constraints and if the Element satisfies all of them, the Kind is added to the given Element. This behavior is depicted in fig. \ref{fig:kind-recognition}.

\begin{figure}
  \centering
  \includegraphics[height=7.5cm]{images/"visualization - kind-recognition"}
  \caption{TransitionKindProcessor detects a GO that satisfies constraints and adds KindElement}
  \label{fig:kind-recognition}
\end{figure}

The similarity To the syntax recognition part means that it can actually replace this part, which is done in this Concept. Instead of checking all constraints themselves, syntax recognition processors, only just check if a given GO has a certain Kind and if this is true, the corresponding DisplayManager is added. As shown in figure \ref{fig:Transition-Kind-Recognition}, when a GO, representing a Transition is added, the TransitionKindProcessor, first adds a Kind with the value Transition, the TransitionRecognitionProcessor, then finds the GO that now has a Transition Kind and adds the corresponding Transition DisplayManager.

\begin{figure}[ht]
\centering
\includegraphics[height=11cm]{images/"visualization - Transition-Kind-Recognition"}
\caption{TransitionRecognitionProcessor adds DM after the TransitionKindProcessor has added a Transition Kind}
\label{fig:Transition-Kind-Recognition}
\end{figure}

with the definition of a convenience member function that checks if an Element has a given Kind, the Place incoming transition invariant can now be rewritten as follows: 
\begin{lstlisting}[language=OCL]
  context PlaceDM: 
  inv:  self.me.incoming = 
            self.go
                .allRelatedTo(ConnectionEnd, 
                    rel | rel.isEndConnection)
                .select(go | go.shape is Line)
                .collect(go | go.relatedFrom(ConnectionEnd))
                .select(go | go.hasKind(Transition))
                .collect(go | go.dm.me.ref())
\end{lstlisting}


It is important to note that the Kind system isn't exclusive to GOs that have an abstract representation. For GO's with a corresponding DisplayManager, the type of this DM can just be checked to find out if it connects a model element one is interested in. But the Kind System can be used to recognize all sorts of patterns in the Graph, that would otherwise have to be checked repeatedly.

\section{Plugins}

While the invariant for incoming transitions, is sufficient for an initial example, it has two mayor flaws. Firstly, it is not checking if the connecting line has an arrow end and secondly only lines that are drawn from the Transition towards the Place, are treated as incoming lines, which is defined by the isEndConnection attribute of the ConnectionEnd relation. Fixing these issues in form of an invariant, would be to verbose vor such a common Pattern.

For this reason, a Plugin system is introduced. Plugins are predefined Processors that process specific parts of the Hypergraph. These processing areas are not as integral as the ones solved by the core processors and thus are opt-in. Furthermore certain plugin processors, can be configured which allows them to be suited to certain tasks. One Example of such a plugin would be the TransitionProcessor. It looks for line GraphicObjects and checks if they connect two other GOs. If that's the case, the Processors adds either a TransitionTo relation or a TransitionBetween relation (fig. \ref{fig:transition-plugin}), depending on if the line's ArrowEnds indicate a directed or undirected Transition. Adding this Plugin allows for a final rewrite of the incoming transitions invariant:

\begin{lstlisting}[language=OCL]
  context PlaceDM: 
  inv:  self.me.incoming = 
            self.go
                .allRelatedTo(TransitionTo)
                .select(go | go.hasKind(Transition))
                .collect(go | go.dm.me.ref())
\end{lstlisting}

\begin{figure}
\centering
\includegraphics[height=7.5cm]{images/"visualization - transition-plugin"}
\caption{TransitionProcessors adds TransitionTo relation on detecting a directed line}
\label{fig:transition-plugin}
\end{figure}

\subsection{Label Processor}




\section{Metamodel}


   


\begin{itemize}
  \item abstraction (implemented with extension functions)
  \item OCL like syntax to navigate tree
  \item Graph Transformations 
  \item forcing the user to draw clear graphs
  \item transformation precondition heavy, post condition easy
\end{itemize}
\chapter{Prototype}

\begin{itemize}
  \item kotlin
  \item guice
  \item free maker
  \item internal kotlin dsl
\end{itemize}


standard language concepts not implemented 

\section{Example 1: Petri Nets}

\section{Example 2: State Charts}

\chapter{Evaluation}
\label{ch:evaluation}
One of the most important parts of a Design Science Research process is the evaluation. The developed artifact has to be analyzed and it has to be established, how well the defined goals are realized by the artifact.

The main criteria that has to be evaluated, is the applicability of the developed artifact. Meaning, it has to be analyzed how well the developed artifact satisfies the defined goal. To this end it seems worthwhile to evaluate the prototypes performance as well as its developer usability. 

\section{Performance}
To provide performance optimization possibilities, \textsc{CouchEdit} was developed with the \texttt{Diff} system in mind. \texttt{Diffs} provide the possibility to calculate changes, without the need for reevaluating the complete hypergraph. On the flipside, this means that all possible states of the graph have to be minded. Using this \texttt{Diff} based approach was evaluated as laborious and error prone, but proved invaluable to improve performance of certain processing tasks \cite{nachreiner_couchedit_2020}. Nachreiner's test results showed that especially language specific processing tasks only took up a small amount of the complete processing time. On basis of this result it was decided, that the developed artifact, primarily concerned with language processing, can produce components that reevaluate the complete graph on every change. Plugin processors, such as the \texttt{LabelProcessor} that is expected to cause high load, are still implemented on application level and thus can make use of the performance benefits granted by the \texttt{Diff} system. 

The actual performance overhead, introduced by this artifact could not be analyzed because of time constraints. Thus, future work would have to evaluate this factor. Furthermore it could be explored if \texttt{Processors}, utilizing the \texttt{Diff} based approach, can be generated from the metamodel, therefore reintroducing the performance benefits. In the current architecture, a \texttt{DM Processors} is triggered every time a change is published, so it can reevaluate the graph. Therefore, a single change done in the frontend will often cause each \texttt{DM Processor} to trigger several times. The metamodel introduces a more structured ordering of \texttt{Processors}. This could be used to split up \texttt{Processors} into sub groups. This way \texttt{Processors} in the same group will calculate changes until no \texttt{Processor} can produce new changes. All changes calculated are then bundled and passed to the next group. This could reduce the number of times a language specific \texttt{Processor}, which reevaluates the complete graph, is triggered.  

\begin{figure}
\centering
\includesvg[width=\linewidth]{images/"component - sub-groups"}
\caption{\texttt{Processors} split into subgroups, where one group is only activated if the previous group has finished processing.}
\label{fig:sub-groups}
\end{figure}

\section{Usability}
Developer usability ask the question on how easy it is to use a tool. The main focus of this research was to develop an artifact that simplifies the process of implementing new modeling syntax configurations for \textsc{CouchEdit}. To this end, it has to be evaluated how well the developed artifact achieves this goal. To find a comprehensive answer to this question, a user study is required. Given the state of the developed artifact as well as the time required to conduct a survey, this goes far beyond the projects scope. Therefore, alternative metrics have to be dissected, in order to determine an inclination regarding this question. 

One metric to highlight is conciseness. As an abstraction of the \textsc{CouchEdit} architecture, the developed artifact should be able to implement similar concepts in less lines of code. The example implementations reflect this. The Petrinets implementation is \petriConfigLoC lines long and produces \petriGeneratedLoC lines of code. Equally, the Statecharts implementation consists of \stateConfigLoC and generates \stateGeneratedLoC lines. In both examples, this means, on average of over 6 lines of source code are generated per line of the configuration. Of course, the code generator most likely generates code that is more verbose than an equivalent written by a developer. therefore this metric is flawed and only serves as a suggestion for the possible usability. 

To further reinforce claims of conciseness, the Statecharts example (\Cref{sec:state-impl}) was modeled as closely as possible after the Statecharts configuration implemented in \cite{nachreiner_couchedit_2020}. An exact replica of the modeling syntax implemented by Nachreiner is not possible. The abstract syntax metamodel defined by them, utilizes \texttt{Relations}, which the AbstractMM defined in \Cref{sec:abstract-syntax} does not support. This results from the close orientation towards Ecore. Furthermore, the \texttt{ConcreteMM} is opinionated in its approach to define configurations and thus, can naturally not provide the same flexibility as an application level implementation has. Nonetheless, the application level implementation is 1900 lines long \cite{nachreiner_couchedit_2020}, and was realized here in \stateConfigLoC lines. This is primarily contributed to the fact that implementations using the metamodel do not have to mind every possible state of the graph, as well as the reduction of a lot of boilerplate code.

While the artifact abstracts away from the implementation details of \texttt{Processors}, the developer still requires certain knowledge about the framework. Most prominently a developer has to understand the hypergraph. Constraints and rules usually require querying of the hypergraph, therefore it is inevitable to know the existing element types as well as possible Relations and their meaning. Furthermore, it has to be understood which changes, different plugins apply to the graph, as well as what functions are available. Nonetheless, the amount of knowledge required, is still reduced.   

\subsection{plugins}
The plugin system proved as a worthwhile addition to the architecture. The example language configurations, described in \Cref{sec:example-configs} show of how complex tasks can be trivialized if the correct plugin is provided. With the plugin system, new \texttt{Processors} can be added to the library of existing processors, should problems arise that are difficult to solve within the generalized metamodel architecture. Plugins are developed on the application level an thus have access to \textsc{CouchEdit}'s complete architecture. On the other hand, this means that developing plugins requires knowledge over the complete \textsc{CouchEdit} framework, which the developed artifact is trying to abstract. Therefore plugins should be implemented as general as possible, so that they can be reused as much as possible. Therein lies the challenge of the plugin system. The usefulness of a plugin depends on how well it can be adapted to different modeling syntaxes. This means that plugins have to be developed with great care. In the example Statecharts implementation (\Cref{sec:state-impl}), the \texttt{LabelProcessor} prototype failed to solve all labeling requirements imposed by the syntax. This demonstrates the volatility of hastily developed plugin. On the other hand, the ConnectionPlugin, thanks to its simplicity was able to provide a concise solution for the given problem.

\section{Discussion of DisplayClasses}
During initial evaluation of the architecture proposed by Fondement and Baar \cite{fondement_making_2005} it was decided to not adapt the concept of \texttt{DisplayClasses}, proposed by the Authors. This decision was made, because their implementation details were not certain. It was estimated, that \texttt{DisplayClasses} would introduce further complexity without providing any significant advantages. with regards to the developed artifact, this turned out as mostly true. Nonetheless, there are certain scenarios as well as considerations towards future work that could make an implementation utilizing \texttt{DisplayClasses} relevant.  

\comment{Something something stricter concrete syntax with (1..1) name for example}


Another scenario that could make \texttt{DisplayClasses} relevant is concerned with syntax feedback in the frontend. \textsc{DiaGen} marks graphical element in a different color, if they are part of a syntactic correct concrete representation \cite{minas_concepts_2002}. Such a feature could be conceivable in future iterations of frontends for \textsc{CouchEdit}. With the pattern system developed in this thesis, there does not exist a direct connection between all \texttt{GOs} and the abstract representation they are part of. Using \texttt{DisplayClasses}, every \texttt{GO} has a direct connection to the abstract representation they are mapped to. Thus, \texttt{DisplayClasses} would trivialize implementation of a feature similar to \textsc{DiaGen}'s.



\section{\comment{Something Something TGG Discussion}}
\comment{Ich sollte wohl irgendwas dazu schreiben, warum ich keinen klassischen TGG approach gewählt hab, aber ich weiß nicht was} 


% \begin{itemize}
%   \item suggestions
%   \item syntactic correctness
%   \item plugin processor needed for specialized tasks
%   \item plugins need to to be well designed (to specialized\\
%         and is general applicability drops,badly designed config model and its a hassle to use )
% \end{itemize}


\chapter{Conclusion}



\begin{itemize}
  \item develop functional DSL
  \item check performance loss
  \item usability test especially in terms of complex graphic objects
  \item abstract syntax correctness and feedback to user
\end{itemize}








% Abbildungsverzeichnis
\cleardoublepage % sonst stimmt die seitennummer im TOC nicht
\phantomsection
\addcontentsline{toc}{chapter}{Abbildungsverzeichnis}
\listoffigures


% Tabellenverzeichnis
\cleardoublepage % sonst stimmt die seitennummer im TOC nicht
\phantomsection
\addcontentsline{toc}{chapter}{Tabellenverzeichnis}
\listoftables


% Bibliographie (am besten mit Bibtex oder Biber)
\cleardoublepage % sonst stimmt die seitennummer im TOC nicht
\phantomsection
\addcontentsline{toc}{chapter}{Literaturverzeichnis}
\printbibliography

% Declaration of Independance
\clearpage
\thispagestyle{empty}

\noindent Name: \fullname \hfill Matrikelnummer: \matrikelnummer \vspace{2cm}

\minisec{Erklärung}

Ich erkläre, dass ich die Arbeit selbständig verfasst und keine anderen als die angegebenen Quellen und Hilfsmittel verwendet habe.\vspace{2cm}

\noindent Ulm, den \dotfill

\hspace{10cm} {\footnotesize \fullname}





\end{document}

% !/ignore-section
