\chapter{Design Concept}



\section{Petri Nets Syntax}


\begin{figure}
  \centering
  \includegraphics[width=.7\linewidth]{images/"csd - petrinet-metamodel"}
  \caption{Metamodel for a simple petri net abstract syntax}
  \label{fig:petrinets_metamodel}
\end{figure}

\begin{figure}[h!]
  \centering
  \begin{subfigure}[t]{.4\textwidth}
    \centering
    \includegraphics[width=.9\linewidth]{images/"petrinet - example"}
    \caption{concrete syntax}
    \label{subfig:petriconcrete}    
  \end{subfigure}
  \begin{subfigure}[t]{.45\textwidth}
    \centering
    \includegraphics[width=\linewidth]{images/"csd - petrinet-example"}
    \caption{abstract syntax}
    \label{subfig:petriabstract}    
  \end{subfigure}
  \caption{concrete and abstract representation for a simple Petri net example}
  \label{fig:petrinets_metamodel}
\end{figure}

   


\begin{itemize}
  \item abstraction
  \item ocl like syntax to navigate tree
  \item Graph Transformations 
  \item forcing the user to draw clear graphs
\end{itemize}


\section{Displaymanagers}
A primary concern that arises when designing a metamodel for CouchEdit, is the question of how to dynamically create abstract syntax elements when a pattern in the concrete syntax is detected that has an abstract representation. For this, the designed architecture draws inspiration from F. Fondement and T. Baar \cite{fondement_making_2005}. As described in section \ref{sec:fondement} the Authors propose to link concrete and abstract syntax with DisplayManagers. This idea can be implemented in CouchEdit.


\section{Abstract Syntax}

\section{Type System}
\begin{itemize}
  \item constraints
\end{itemize}

Transformations

\begin{itemize}
  \item DC processors create
  \item DM processors sync
\end{itemize}

\section{Plugins}
\begin{itemize}
  \item implement needed processes
  \item can need specialized configuration
\end{itemize}
\subsection{Label Processor}


\section{Graphic Object Definitions}