\chapter{Related Work}
\label{ch:related_work}
\comment{chapter auch noch so ziemlich nicht existent}

There are multiple projects that implement a form of relaxed conformance editing. But only few of these projects actually define languages that can be utilized to configure different modeling syntaxes. 




\section{Triple Graph Grammars}
\comment{i dont konw...}


\section{Making Metamodels Aware of Concrete Syntax}
\label{sec:fondement}
in their work, Making Metamodels Aware of Concrete Syntax \cite{fondement_making_2005}, F. Fondement and T. Baar argue that, while abstract syntax definitions are standardized, most language specifications keep the concrete syntax informal. To solve this problem, they propose an approach to defining the concrete syntax and how to link it to the abstract representation.

For this, the authors complement every class of the abstract syntax with a corresponding display scheme. This display scheme is compose of two parts, an iconic and a constraining part. The iconic part defines a set of \texttt{DisplayClasses}, these \texttt{DisplayClasses} group Graphical Objects together into visual representation object. On the other hand, the constraining part links these \texttt{DisplayClasses} to an abstract syntax element. This link is realized using \texttt{DisplayManagers}. A \texttt{DisplayManager} serves as connection between exactly one model element of the abstract syntax and one display object and has the task of syncing the abstract to the concrete representation. An example of this architecture is depicted in \Cref{fig:fondement_dm}, for a Petrinet place element. The graphical primitives, a place is composed of, are mapped to a place \texttt{DisplayClass}, to build a places iconic part. This iconic representation is then attached to a place model element, using a place \texttt{DisplayManager}.


\begin{figure}[H]
  \centering
  \includesvg[width=\linewidth]{images/"csd - fondement-example"}
  \caption{Example representation of a Petrinet place element}
  \label{fig:fondement_dm}
\end{figure}

A DisplayManager has to keep abstract and concrete representation in sync, for this Fondement and Baar utilize OCL invariants, which are defined on the DisplayerManager. For example, an invariant to sync the name of place display schemes could look as following:

\begin{lstlisting}[language=OCL,captionpos=b,caption={OCL Invariant that syncs the name attribute of \texttt{DisplayClass} and model element.}]
context PlaceDM
inv: self.me.name->exists() implies
        self.me.name = self.vo.name.text
\end{lstlisting}

The Authors keep open, how the mapping from graphical primitives to a display object could be implemented. while the metamodel proposed in this thesis does not introduce an extra layer of abstraction in form of these \texttt{DisplayClasses}, it still is inspired heavily by Fondement's and Baar's work. Especially the constraining part that utilizes \texttt{DisplayManagers} to sync abstract and concrete syntax, served as primary inspiration for the proposed architecture.

In the work "Correctly defined concrete syntax" \cite{baar_correctly_2008}, Baar furthermore provides an algorithm to check syntactic correctness of the concrete representation. While checking the concrete syntax for correctness is out of scope of this work, a future implementation would have to provide the user with syntax checking capabilities, to make sure that the model, designed in the editor is syntactically correct.


\section{DiaGen}
\label{diagen}
\textsc{DiaGen} is a project first introduced by Minas and Viehstaedt \cite{minas_diagen_1995}. It is a system that allows for the specification and generation of modeling editors. \textsc{DiaGen} supports \emph{free-hand editing}, where the user can manipulate the graphical representation directly. This is the same editing mode that \textsc{CouchEdit} supports \cite{nachreiner_couchedit_2020}. In \textsc{DiaGen}, a diagram consists of a finite set of diagram components \cite{minas_concepts_2002}. Each component is defined by its attributes. Furthermore, each component has a set of \emph{attachment points}. Attachment points define areas at which different diagram components can be connected. \textsc{DiaGen}'s architecture generates a hypergraph from the visual diagram consisting of these diagram components. This hypergraph is furthermore reduced using a context-free grammar to receive the abstract representation. \textsc{DiaGen} also supports automatic layouting and user feedback on diagram parts that are not correct under the defined diagram language \cite{minas_concepts_2002}.

\textsc{DiaGen} provides a textual language as well as a GUI tool, which can be used to specify diagram languages. 


\comment{kinda short at the moment}
