\chapter{Fundamentals}
\label{ch:fundamentals}

This chapter lays out the knowledge foundation required in later sections. First, the methodology that guided this research will be described. Following that, modeling language fundamentals and two modeling language notations are represented. Finally, a comprehensive overview of the \textsc{CouchEdit}'s architecture and its features is provided. 

% this chapter lays out the knowledge foundation required in later sections. First, we give an overview of the methodology, we used to guide this research. Following that, we introduce modeling language fundamentals and two modeling language notations we will use later on in the work to explain certain concepts. Finally, we provide a comprehensive overview of \textsc{CouchEdits}'s architecture and features.


% First, an outline of the applied research process is given. Following that, a comprehensive view of the \textsc{CouchEdit} architecture is given. This will be needed later on to understand certain design decisions. Thereafter, a foundation of modeling languages is laid out, as well as some modeling syntaxes. Finally, a summary of related work and their impact on this thesis will be presented. 

\section{Methodology}
This research strived to develop a new metamodel suitable for the \textsc{CouchEdit} framework. Therefore, it was conducted following the Design Science Research (DSR) approach. According to V. Vaishnavi et al., a design science research process consists of five steps \cite{vaishnavi_design_2004}.

\subsection{Problem Awareness}
The first step of design science research is the identification of existing problems. As specified in \Cref{sec:problem_statement}, it was identified that the current \textsc{CouchEdit} implementation lacks a developer-friendly way to configure it for different modeling syntaxes and that it is unclear how to design a metamodel for this architecture.

\subsection{Suggestion}
% todo 
With a clear definition of the problem, objectives can be proposed which have to be achieved in order to solve the identified problem. The first objective of this research is to develop a metamodel applicable to\textsc{CouchEdit}'s architecture. To be more precise, a metamodel is to be designed that can be used to specify model syntax definitions that map concrete graphical syntaxes to Abstract syntax models and is applicable to \textsc{CouchEdit}'s architecture. The second objective is to implement a prototype that provides proof of concept for the designed metamodel's applicability.


\subsection{Development}
The primary goal of design science research is the development of artifacts.
The first artifact to be developed in this research will be a metamodel that can be used to define new modeling syntaxes for a relaxed conformance editor and applies to \textsc{CouchEdit}'s architecture. The second artifact that is to be developed is a prototypical code generator that can translate the developed metamodel into a \textsc{CouchEdit} implementation, which will be able to process the defined modeling syntax. Furthermore, it was decided to implement a simple DSL that eases the process of defining instances of the given metamodel.

To this end, the first sub-step of the development stage has to be a comprehensive analysis of \textsc{CouchEdit}'s architecture. In his work, Nachreiner describes in detail which modeling features the framework covers and how they are implemented \cite{nachreiner_couchedit_2020}. The goal is to develop an approach that can streamline defining modeling syntaxes while adhering to \textsc{CouchEdit}'s specification. Related work defined in \Cref{ch:related_work} serves as a basis from which a first suitable metamodel can be derived.

In the next sub-step, the code generator will be implemented. This is an iterative step. First, an implementation will be developed based on the designed metamodel. This implementation should reveal further requirements imposed by \textsc{CouchEdit}'s architecture (RQ2). The metamodel then has to be adjusted to account for these requirements, which in turn requires changes of the implementation. This iteration has to be repeated until no further requirements are deductible. Because of time constraints, it is not possible to implement all of the metamodel's features. Instead, the prototype covers a minimal feature set that still suffices to demonstrate the metamodel's applicability for selected modeling syntaxes.
% todo: maybe define selected modeling syntaxes here already

\subsection{Evaluation}
Now that the desired artifacts are fully developed, it has to be evaluated how well they do their designated tasks. As the primary artifact of this research, the metamodel has to be evaluated in terms of its applicability to its designated domain. To this end, different modeling syntaxes will be implemented using the developed code generator. This will serve as a demonstration of the developed artifacts and should highlight areas in which the metamodel's design excels, as well as design flaws and limitations (RQ3). If time allows, potential flaws can be addressed by returning to the development phase and revising the artifacts. Otherwise, flaws are to be highlighted so that future work can address them.

\subsection{Conclusion}
The conclusion stage marks the end of a DSR, and the results have to be communicated. This written thesis as well as all source code and documentation will serve as means to communicate the results of this research.


\section{Modeling Languages}
\label{sec:modeling-languages}
This thesis discusses modeling syntaxes and how \textsc{CouchEdit}'s architecture handles them. To this end, this section introduces the Petri nets and Statecharts syntaxes. They will serve as examples to demonstrate various problems and applicability of the developed artifacts. 

% \subsection{Graphical Language Theory}
\subsection{Petri nets Syntax}
\label{sec:petrinets}
Petri nets are easy to understand and provide simple abstract and concrete syntax. Thus they serve as a suitable tool for explaining concepts in the following chapters. \Cref{fig:petrinets_metamodel} shows the abstract syntax metamodel of the used Petri nets syntax. A Petri nets conceptual representation consists of \emph{places} and \emph{transitions}. While Both possess a name, places furthermore have a token count. Each place can have a variable amount of incoming and outgoing transitions, while transitions can have any amount of incoming and outgoing places. 

\begin{figure}[H]
  \centering
  \includesvg[width=.5\linewidth]{images/"csd - petrinet-metamodel"}
  \caption{Metamodel for a simple Petri nets abstract syntax}
  \label{fig:petrinets_metamodel}
\end{figure}

The graphic primitives, a concrete representation of Petri nets is composed of, are listed in \Cref{tab:petri-primitives}. Usually, circles without fill color represent places, while slender rectangles represent transitions. The number of small black circles inside a place represents this place's token count. Furthermore, places and transitions possess a label close to their bounding box, that determines their name. Lastly, places and transitions have directed connection lines between them. Each connection points from a place towards a transition or from a transition towards a place. A connection marks the target element as an outgoing reference for the source element and vice versa. \Cref{fig:petrinets_example} shows a simple concrete instance and the corresponding abstract representation.


% Each connection points from a place or transition towards an element of the opposite type and represents an outgoing connection for the source element and an incoming connection for the target element. A simple concrete instantiation and its corresponding abstract representation is shown in \Cref{fig:petrinets_example}.

\begin{table}[ht]
  \centering
\begin{tabular}[width=.1\linewidth]{| Sc | Sc | Sc | Sc | Sc |}
  \hline
  Place & Transition & Token & Label & Connection 
  \\
  \hline
  \includegraphics[width=.1\linewidth]{images/"petrinet - place"} 
  & 
  \includegraphics[width=.1\linewidth]{images/"petrinet - transition"} 
  & 
  \includegraphics[width=.1\linewidth]{images/"petrinet - token"}
  & 
  \includegraphics[width=.1\linewidth]{images/"petrinet - label"}
  & 
  \includegraphics[width=.1\linewidth]{images/"petrinet - connection"} 
  \\
  \hline
\end{tabular}
\caption{graphic primitives used to describe Petri nets}
\label{tab:petri-primitives}
\end{table}

\begin{figure}[ht!]
  \centering
  \begin{subfigure}[t]{.35\textwidth}
    \centering
    \includesvg[width=.9\linewidth]{images/"petrinet - example"}
    \caption{concrete syntax}
    \label{subfig:petriconcrete}    
  \end{subfigure}
  \begin{subfigure}[t]{.4\textwidth}
    \centering
    \includesvg[width=\linewidth]{images/"csd - petrinet-example"}
    \caption{abstract syntax}
    \label{subfig:petriabstract}    
  \end{subfigure}
  \caption{concrete and abstract representation for a simple Petri nets example}
  \label{fig:petrinets_example}
\end{figure} 

\subsection{Statechart Syntax}
\label{sec:statecharts}
As a second example, Statecharts are introduced here. The term Statecharts was first coined in 1987 by D. Harel \cite{harel_statecharts_1987}. Statecharts are an extension of state machines. Their primary feature is that each state can have itself sub-state machines. \Cref{fig:statechartmm} describes the abstract syntax metamodel used in this thesis. This abstract syntax does not represent a complete specification of the Statecharts syntax, rather a subset than can highlight specific areas of the implementation. Statecharts define a more complex syntax than Petri nets mentioned above and thus will show the developed artifacts applicability towards more complex modeling languages.

\begin{figure}
\centering
\includesvg[width=.8\linewidth]{images/"csd - new-statechart-metamodel"}
\caption{Abstract Metamodel of Statecharts}
\label{fig:statechartmm}
\end{figure}


A Statechart is composed of \emph{StateElements} and \emph{Transitions}. Transitions might connect StateElements. Each Transition has exactly one source StateElement and one target StateElement. StateElements can either be \emph{PseudoStates} (e.g., initial state, choice, etc.) or normal \emph{States}. A State can either possess a name, a name and, a sub-state machine, or multiple \emph{Regions}. Each Region contains one state machine. Regions are marked by splitting up States using dashed lines. 

% todo: write some more

\begin{figure}
\centering
\includesvg[width=.7\linewidth]{images/"visualization - statechart-example"}
\caption{Example Statechart}
\label{fig:state-example}
\end{figure}

% \begin{table}[ht]
%   \centering
% \begin{tabular}[width=.1\linewidth]{| Sc | Sc | Sc |}
%   \hline
%   Simple State & Compound State & Compound State w/ Regions
%   \\
%   \hline
%   \includegraphics[width=.2\linewidth]{images/"visualization - simpleState"} 
%   & 
%   \includegraphics[width=.3\linewidth]{images/"visualization - compoundState"} 
%   & 
%   \includegraphics[width=.35\linewidth]{images/"visualization - region-compound-state"}
%   \\
%   \hline
% \end{tabular}
% \caption{Possible representations for states}
% \label{tab:state-states}
% \end{table}



\section{CouchEdit}
\label{sec:CouchEdit}


% \comment{write basics about CouchEdit and introduce views}

% \begin{figure}
% \centering
% \includegraphics[width=.7\linewidth]{images/"couchedit-main"}
% \caption{Main view of the \textsc{CouchEdit} prototype configured for Statecharts}
% \label{fig:couchedit-main}1
% \end{figure}

% \begin{figure}
% \centering
% \includegraphics[width=.9\linewidth]{images/"couchedit-debug"}
% \caption{Debug view of the \textsc{CouchEdit} prototype, depicting the abstract syntax graph}
% \label{fig:couchedit-debug}
% \end{figure}




As noted in \Cref{sec:problem_statement}, the \textsc{CouchEdit} framework builds on the ideas of Van Tendeloo et al. \cite{van_tendeloo_concrete_2017}. In their paper, the authors criticize that classic approaches to modeling frontends usually create one monolithic construct, which handles both the graphical model display and the corresponding abstract syntax. This creates tight coupling between the two, which causes low flexibility of the UI and high effort requirements to implement new features. 

The authors propose to solve this by defining two metamodels, one describing the graphical syntax and one describing the abstract syntax. The render syntax metamodel specifies graphic primitives that a given graphical modeling syntax uses. The frontend manages instances of this metamodel, and a user can manipulate this concrete representation directly by adding, deleting, or modifying elements in the UI (if the frontend allows changes). On The other side stands the abstract syntax metamodel, it describes a modeling syntax's conceptual representation. The backend manages instances of this metamodel. A further metamodel, called \emph{concrete syntax metamodel}, can then be defined to connect the render and abstract syntax metamodels. This concrete syntax metamodel specifies a connection that maps the concrete graphical representation onto an abstract representation (\Cref{fig:transmm}).

% To then connect these two metamodels, a further metamodel, called concrete syntax metamodel, can be defined. This concrete syntax metamodel specifies a connection that maps the concrete graphical representation onto an abstract representation (\Cref{fig:transmm}). 

\begin{figure}
  \centering
  \includesvg[width=.7\linewidth]{images/"presentation - transfere-metamodel"}
  \caption{Depiction of \textsc{CouchEdit}'s approach to separating graphical and abstract representation}
  \label{fig:transmm}
  \end{figure}

Following this architecture, \textsc{CouchEdit} separates the frontend from the  abstract syntax model. At its core, the frontend is a simple vector drawing application that implements further functionality for diagrams. A Backend then has the responsibility to analyze these vector graphics and map them onto a corresponding abstract syntax representation. For this, the backend utilizes a set of independent components, that each is responsible for detecting different aspects of the diagram. Each Component contains a separate state. When a component makes changes to its state, it also publishes these changes so that other components, interested in these changes, can use them to calculate follow-up changes.  

This architecture has the advantage of being very detangled. Therefore, it is possible to improve components in isolation from the rest of the system. The clear separation of concerns also makes it easier to understand which component is responsible for what task. Furthermore, components can be swapped out depending on the tasks the configured modeling syntax requires.
% what tasks are needed for the configured modeling syntax.



\subsection{Data Model}
\textsc{CouchEdit} uses a hypergraph of Elements and Relations to represent its data. Each component has its own instance of this graph that only contains the graph area relevant to this component. A component can manipulate its graph and make these changes known to the system so that other components can use them.


\subsubsection{Elements}
\emph{Elements} are the base type of \textsc{CouchEdit}'s hypergraph. Every type of node in this graph inherits from the Element type. As all hypergraph instances should be independent of each other, it is impossible to access Elements via their object reference. Thus every Element has a unique id to identify it across multiple graphs reliably. Elements also have a \emph{probability} attribute that denotes how probable it is that this Element results from a correct interpretation of the given hypergraph. The probability attribute can also be set to \emph{Explicit}, which indicates that Processors should prioritize the corresponding Element regardless of the probability value. Furthermore, Element subtypes can introduce further attributes that contain important information for the given type.


% This probability object can be set to \emph{Explicit}, which means that either this Element describes a factual truth, or it was marked by the user as not to be changed. 

\subsubsection{Relations}
\label{sec:relations}
\emph{Relations} represent edges in the hypergraph. Each Relation connects a set of source Elements to a set of target Elements. They also inherit from the Element type and thus can recursively be connected with relations themselves. The connected vertices are referenced using the \emph{ElementReference} class. This class holds an Element's unique id and its type, which then can be used to find a referenced Element in the hypergraph. \Cref{fig:relations} shows the implementation of Relations. 

\begin{figure}[ht]
  \centering
  \includesvg[width=.8\linewidth]{images/"csd - relation"}
  \caption{Class diagram of Relations}
  \label{fig:relations}
\end{figure}

 While Relations can connect multiple source and target vertices, the most common Relation type is the \emph{OneToOneRelation}, it describes a connection between exactly one source and one target Element. A typical example of such an OneToOneRelation is the \emph{Contains} relation, which defines that one Element encloses another.  


\subsubsection{GraphicObjects}
\emph{GraphicObjects} (GOs) represent atomic elements that the user interacts with via the Frontend. When designing an Editor, an important question is how granular a singular graphic object should be. On a basic level, GOs can consist of either pixels or vectors. A pixel-based Approach would require a visual recognition preprocessor that composes the state of pixels into comprehensible graphic primitives. While such form of image recognition is outside \textsc{CouchEdit}'s scope, similar projects such as \textsc{FlexiSketch} \cite{wuest_flexisketch_2015} show this approaches's application.

On the other hand, most modeling approaches with abstract syntax processing build specialized graphic elements to represent a specific element of the abstract syntax. Therefore mapping the concrete syntax to an abstract representation is made easy. However, as a result, these graphic elements are stringent in their graphical representation.

\textsc{CouchEdit}'s goal is to allow users to construct diagrams using elementary, "multi-purpose" graphic objects as well as specialized elements (if required by the modeling language) \cite{nachreiner_couchedit_2020}. To this end, \textsc{CouchEdit} uses a render metamodel (RenderMM). It defines all building blocks that are usable in a concrete representation. \textsc{CouchEdit} builds its RenderMM upon the Classification of Costagliola et al. \cite{costagliola_classification_2002}. This classification defines a graphic element on its lowest level as a primitive graphic shape (e.g., circle, line). Via sub-classing, these primitive GOs can then be extended to form specialized, \emph{complex graphic objects}.

In his work, The “Physics” of Notations: Toward a Scientific Basis for Constructing Visual Notations in Software Engineering \cite{moody_physics_2009}, Moody specifies eight variables that can be applied to a single graphic primitive. These attributes are separated into factors that directly influence the relations between GOs:  Position (horizontal and vertical), shape, size, orientation, and attributes that only serve as visual hints and influence a single GO at a time: color, brightness, texture. \textsc{CouchEdit} realizes this separation as well. Factors with spatial influence are contained in the direct info of the shape. While purely visual attributes are externalized into \emph{AttributeBags}. An AttributeBag is a separate Element attachable to GraphicObjects that contains secondary information about the given GO.

\subsubsection{HotSpotDefinitions}
\label{sec:hotspotdefinitions}
Some Modeling Notations define relevant zones in a graphical notation, which are not directly modeled by a GraphicObject and thus are not directly visible to the user but exist due to the visual composition. For example, in \cite{bottoni_suite_2004}, attachment points have been identified as such a relevant zone. An attachment point defines an area around a GraphicObject where a connection line can be "attached". \textsc{CouchEdit} implements this in a more general form through \emph{HotSpotDefinitions}. A HotSpotDefinition is a relation that, similar to GraphicObjects, has a shape and can thus be a target and a source of spatial relations.

One of the primary applications of HotSpotDefinitions is in the form of \emph{CompartmentHotSpotDefinitions} (CompHSDs). Compartments split a GraphicObject into multiple logical sub-areas. This is used in multiple advanced notations. For example, the Statechart syntax discussed in \Cref{sec:statecharts}, where a State can have multiple Regions marked by dashed lines. 

\subsection{Application Structure}
\textsc{CouchEdit} builds around independent components, called \emph{Processors} that can be enabled or disabled as needed in a given use case. To this end, the centerpiece of \textsc{CouchEdit}'s architecture is the \emph{ModificationBusManager}. Every Processor (including the frontend) can be connected to this bus manager (\Cref{fig:processors}). If one Processor publishes a hypergraph change, the ModificationBusManager propagates this change to all connected Processors. Each Processor can specify which Element types it is interested in and will only receive updates for Elements that match one of those types. 

To communicate these changes in the system, \textsc{CouchEdit} uses \emph{ModelDiffs}. Each Diff represents an add, change, or delete action of an Element in the Hypergraph. This allows Processors to calculate correct graph changes based on the incoming changes instead of reevaluating the complete graph. This was added to the system because Nachreiner suspected that reevaluating the complete state would cause much overhead \cite{nachreiner_couchedit_2020}. It turned out that Processors which have to calculate spatial relations significantly profited from diff based calculations  \cite{nachreiner_couchedit_2020}. However, calculating the correct follow-up state based on a set of Diffs turned out to be highly error-prone. Thus it was suspected that it could be beneficial to reevaluate the complete graph when processing lighter tasks, such as Abstract syntax processing \cite{nachreiner_couchedit_2020}.

\subsubsection{Core Processors}
\label{sec:core-processors}
\textsc{CouchEdit} implements a set of Processors that are integral to most modeling syntaxes. These so-called \emph{core Processors} currently are: 
\begin{description}
  \item[SpatialAbstractor] This Processor has the responsibility to calculate how GOs are positioned to each other. Possible position Relations include: \emph{RightOf}, \emph{BottomOf}, \emph{Intersect}, etc..
  \item[ConnectionEndDetector] The \emph{ConnectionEndDetector} finds line GOs that could represent a connection to another GO. It then adds \emph{ConnectionEnd} relations from the line to the other GO. Furthermore, the ConnectionEnd relation has the attribute \emph{isEndConnection} that defines if this relation origins from the ending point of the line.
  \item[Containment] The \emph{ContainmentProcessor} checks if one GO is contained by another. If one GO completely encompasses another GO, the Processor adds a Contains relation from the surrounding GO to the contained one.
\end{description}

\subsection{Services}
\label{sec:services}
Processors may sometimes require to gather additional information. For example it is often needed to find all Elements that are connected to a given Element by a certain Relation type. To this End, \textsc{CouchEdit} provides \emph{Services}. A Services is a separate objects that provide a set of functions. A Service can be required by multiple different Processors and thus must be effectively stateless.

% todo: more



