\chapter{Introduction}
\label{ch:introduction}


Modeling languages have long played a vital role in software engineering. Properly designed models can abstract complex systems and provide a visual aid in understanding them. Furthermore, in the form of the Business Process Model Notation (BPMN), they can define and automate processes. Today, with research around Model-Driven Engineering (MDE), a paradigm centered around models, with the intent to generate code bases and whole systems from them, the importance of models is rising even more.

As these tasks require syntactic correctness of used models, modeling tools become an essential part of an engineer's workflow. In theory, visual modeling tools especially provide an intuitive and user-friendly way to design models. Current graphical modeling tools often constrain users in unintuitive ways and deliver subpar user experience (UX), commonly stemming from a tight coupling between a modeling tools user interface (UI) and the underlying model. As the model's syntax usually is inflexible, the UI must make restrictions to adhere to this syntax, often creating issues for the user. Typically, this can manifest as connections only being drawable between two existing states or the undesired deletion of child nodes coupled to the removal of the parent.

To amend these usability woes, Nachreiner proposed a novel modeling framework called \textsc{CouchEdit} \cite{nachreiner_couchedit_2020}. \textsc{CouchEdit} is categorized as a relaxed conformance modeling Framework. This means it can loosen the strictness requirements that are imposed onto the graphic model representation. To this end, the framework decouples user interface and model syntax by introducing different models for both. In the \textsc{CouchEdit} architecture, the user interface, instead of relying directly on the syntax of the model to be designed, uses a render-model that specifies only basic graphic objects. On the other hand, the conceptual representation of the model now stands on its own. A third metamodel then connects these two models. At its core, \textsc{CouchEdit} is a general-purpose framework, rewritable to adhere to any model syntax. Nevertheless, to realize this in the current implementation, the source code must be changed directly, which is error-prone, convoluted, and requires an understanding of \textsc{CouchEdit}'s internal architecture.

Aiming to create a more developer-friendly and flexible way of adapting \textsc{CouchEdit} to different modeling syntaxes, this design research proposes a new metamodel that can create modeling syntax definitions, which are usable by \textsc{CouchEdit}. Furthermore, a conceptual parser is presented, that provides proof of concept on how this newly developed metamodel interacts with the \textsc{CouchEdit} architecture.

\section{Problem Statement}
\label{sec:problem_statement}

A general-purpose framework should be configurable for multiple use cases in its designated domain. \textsc{CouchEdit}, as a general-purpose graphical modeling framework, thus should be configurable for multiple modeling syntaxes. Technically this is possible, as long as one has access to the source code. However, this would mean, every time \textsc{CouchEdit} has to support a new modeling syntax, manual changes in the source code have to be made, and the project has to be compiled from sources. Implementing a configuration parser that can interpret modeling syntax definitions at runtime would thus increase flexibility. Furthermore, a well-designed metamodel could reduce the amount of knowledge required about the \textsc{CouchEdit} framework when configuring new model syntaxes.

As a relaxed conformance editing framework, \textsc{CouchEdit} poses special architectural requirements. It must allow for temporary inconsistencies between concrete syntax (what the user draws) and abstract syntax (what the underlying Model looks like). As the concrete representation does not always have to map to a syntactic correct abstract model, the user has more freedom during the modeling process (e.g., dangling transitions\footnote{Many modeling editors force transition lines to always connect two model elements. If one of the connected elements is removed, the line will also be removed as a result.}).

\textsc{CouchEdit} achieves this by building upon the architecture concept of clear separation between concrete and abstract syntax, proposed by Y. Van Tendeloo et al. \cite{van_tendeloo_concrete_2017}. Internally, \textsc{CouchEdit} builds a hypergraph that maps the given concrete syntax to all possible abstract syntaxes (interpretation of a concrete syntax can be ambiguous, and multiple abstract syntaxes can be possible). A set of processors, connected in a reactive publish and subscribe pattern, are responsible for building this Hypergraph. All processors (and the user interface) are subscribed to a bus (\Cref{fig:processors}). If a change (diff) is published to the bus (e.g., the user adds a node to the concrete syntax), all processors that are subscribed to this type of diff are notified and calculate new resulting diffs, these new diffs are then also published to the bus, and all processors interested in them are invoked as well.

\begin{figure}
  \centering
  \includesvg[width=\linewidth]{images/"component - bus"}
  \caption{Publish Subscribe architecture of \textsc{CouchEdit}}
  \label{fig:processors}
\end{figure}

Some of these processors are needed for every type of syntax model. For example, the spatial processor processes how nodes in the concrete syntax are positioned to each other (above, besides, etc.). Other processors are specific to the given modeling syntax. For example, a Statechart syntax would require a state processor that processes if a given graphical object represents a state (usually true if the given node is a rectangle with rounded edges).

A modeling syntax parser for \textsc{CouchEdit} would have to generate these syntax specific processors while considering multiple design constraints that result from this architecture. Thus a metamodel is needed that can define the desired abstract syntax and specify a mapping from concrete representations to this abstract syntax. While there is ongoing research in the area of relaxed conformance editing and how to link concrete and abstract syntax, the design of such a metamodel does not immediately become apparent.

\section{Purpose of this Study}
The primary purpose of this study was to develop and evaluate a metamodel for the \textsc{CouchEdit} framework. This metamodel is supposed to provide a comprehensive and easy to use way for defining new modeling languages. Furthermore, a prototypical code generator was implemented to evaluate the metamodel's applicability. This code generator can comprehend the newly designed metamodel and translate it into a source code implementation, which the current implementation of \textsc{CouchEdit} can use. Additionally, a simple domain-specific language was developed that eases the process of specifying instances of this metamodel.

This extension of the \textsc{CouchEdit} framework is supposed to improve developer accessibility and framework flexibility. While a code generator means that the system still has to be recompiled for every modeling syntax, it still becomes more comfortable to support new modeling syntaxes as the metamodel abstracts away from the actual source code implementation. It thus requires less knowledge about the \textsc{CouchEdit} framework, as well as fewer lines of code.


\section{Research Questions}
The primary goal of this research was the development of a metamodel for the \textsc{CouchEdit} framework. To this end, the following questions were defined as a means to guide the research.

\begin{description} 
  \item[RQ1:] Is it possible to define a concise metamodel that covers \textsc{CouchEdit}'s feature set?

        If this is not possible, the following question has to be answered.
        \begin{description}
          \item[RQ1.1:] How could the feature set be narrowed down without impacting the framework's capabilities too much?
        \end{description}

  \item[RQ2:] What requirements does the \textsc{CouchEdit} architecture impose on a metamodel?

        The architecture's characteristics and quirks have to be considered when designing the language. For example, the publish and subscribe pattern could make it easy to introduce nonterminating processing chains, which the metamodel should prevent where possible.

  \item[RQ3:] How applicable is the designed metamodel for defining new modeling syntaxes?

        \begin{description}
          \item[RQ3.1] Is it possible to express common modeling syntax concepts clearly and concisely?
          \item[RQ3.2] Are complex concepts still expressible without causing too much convolution?
        \end{description}
\end{description}


% \section{Limitations}
% Because of the tight time constraints imposed by the scope of this project, this thesis does not provide a complete solution for the given problem. Rather the developed artifact presents a cross section of the areas that are of importance. This Thesis does not:

\section{Thesis Structure}
This thesis is organized in the following structure: \Cref{ch:fundamentals} lays out the fundamentals required in the rest of the thesis. This includes a comprehensive summary of the \textsc{CouchEdit} architecture as well as modeling theory. After that, \Cref{ch:related_work} summarizes related work in the relevant fields. Following that, in \Cref{ch:design}, the developed metamodel is described in detail. \Cref{ch:prototype} then presents the prototype implementation. \Cref{ch:evaluation} provides considerations about the evaluation of the artifact, as well as the evaluation results. Finally, \Cref{ch:conclusion} provides takeaways and perspectives on the future of this project.