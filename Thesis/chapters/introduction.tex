\chapter{Introduction}


Modeling languages have long played an important role in software engineering. Well designed models can abstract complex systems and provide visual aid in understanding them. Furthermore in form of the Business Process Model Notation (BPMN) they are used to define and automate processes. Today, research in the area of Model Driven Engineering (MDE), a paradigm centered around models, with the intent to generate code bases and whole systems from them, the importance of models is rising even more.

As these tasks require syntactic correctness of used models, modeling tools become an essential part of an engineer's workflow. Especially visual modeling tools provide in theory, an intuitive and user friendly way to design models. But current graphical modeling tools tend to constraint users in unintuitive ways and deliver sub par user experience (UX). this usually arises from a tight coupling between a modeling tools user interface (UI) and the underlying model. As the model's syntax is usually inflexible, the UI has to make restrictions to adhere to this syntax. this often creates problems for the user, for example connections can only be drawn between two existing states, or deleting a node will result in all its children being deleted as well.

To amend these usability woes, L. Nachreiner proposed a novel modeling framework, called CouchEdit \cite{nachreiner_couchedit_2020}. This framework decouples user interface and model syntax by introducing different models for both. Instead of relying directly on the syntax of the model that is being designed, in the CouchEdit architecture the user interface is using a render model that only consists of nodes that are rendered in the Modeling tool, called concrete syntax. On the other hand, the actual models syntax now stands on its own, called abstract syntax. To translate between concrete and abstract syntax, a syntax metamodel is utilized. CouchEdit at its core was designed to be general purpose, meaning it can be rewritten to adhere to any model syntax. But to realize this in the current implementation, the source code has to be changed directly, which is error prone, convoluted and requires some understanding of CouchEdits internal architecture.

To create a more developer friendly and flexible way of adapting CouchEdit to different modeling syntaxes, this design research aims to propose a new metamodel, that can be used to create modeling syntax definitions which are usable by CouchEdit. Furthermore a conceptual parser is to be developed and implemented, that provides proof of concept on how this newly developed metamodel interacts with the CouchEdit architecture.