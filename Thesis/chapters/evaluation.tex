\chapter{Evaluation}
\label{chap:evaluation}
One of the most important parts of a Design Science Research process is the evaluation. The developed artifact has to be analyzed and it has to be established, how well the defined goals are realized by the artifact.

The main criteria that has to be evaluated, is the applicability of the developed artifact. To measure this, there are multiple possible metrics that can be considered:

\begin{enumerate}
  \item Is the developed artifact easier to learn and understand, than the pre existing approach.
  \item What is the performance cost of the developed artifact.
  \item how concise is the newly developed syntax.
\end{enumerate}

The main point of concern while developing was developer usability. Meaning, how easy it is for a developer to configure new modeling syntaxes for CouchEdit. 

\section{Applicability}
metamodel is TGG + hypergraph deckt alles von Bottoni ab -> Metamodel kann in theory jeden modeling syntax konfigurieren

\begin{itemize}
  \item in terms of complexity, was shown that correct plugin design is important
\end{itemize}


\section{Performance}
To reduce complexity, Annotation



Nachreiner showed in his work \cite{nachreiner_couchedit_2020} that 



\begin{itemize}
  \item ordered Processing seems applicable (core processors first, then plugins, etc.)
  \item way more processing -> performance loss
  \item plugin processor needed for specialized tasks
  \item plugins need to to be well designed (to specialized\\
        and is general applicability drops,badly designed config model and its a hassle to use )
\end{itemize}

